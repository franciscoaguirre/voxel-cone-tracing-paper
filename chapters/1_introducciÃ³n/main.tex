\chapter{Introducción}

% Aquí se motiva el trabajo, se plantea y define el problema, se deja claro cuales son los objetivos (general del proyecto, si correspondiese o si está inmerso en un proyecto de mayor alcance, y los específicos), se plantean los resultados esperados, se establecen resumidamente las conclusiones y se describe la
% organización general del documento.

El problema de la iluminación global en tiempo real ha sido muy estudiado.
Resolverlo es uno de los objetivos largamente buscados de la computación gráfica, debido a que tiene una alta importancia en varias industrias, como la de los videojuegos, la del cine, las simulaciones físicas, entre otros.
Existen varias técnicas para resolver el problema de la iluminación, como path tracing y photon mapping, pero estas fallan en conseguir tiempos interactivos con el hardware disponible.
Se han hecho varias optimizaciones a lo largo de los años para lograr esto, como por ejemplo simplificaciones en la geometría, el uso de estructuras jerárquicas, clustering, entre otras, pero ninguna de estas logró que funcionaran en tiempo real, y sufren de artefactos como el ruido.

El objetivo de este trabajo es aprender sobre y crear una implementación de un algoritmo de iluminación global que logra correr en tiempo real sin necesidad de hardware específico, \textit{voxel cone tracing}.
Este algoritmo fue propuesto por Crassin et al en 2011 \cite{voxel-cone-tracing}.
No sufre de problemas de ruido y provee una buena calidad de imágenes con un rendimiento casi independiente de la complejidad de la escena, debido a que la geometría en sí no se usa en los cálculos de luz, si no una aproximación de vóxeles.
Computa hasta dos rebotes de la luz en su camino desde el emisor hacia la cámara, lo que permite incorporar el componente principal de la luz indirecta.
El algoritmo surge en un contexto en el que el problema de iluminación global en tiempo real no tenía muchas soluciones, y la demanda para la misma estaba creciendo.
Hace uso de varias técnicas que estaban disponibles, la representación de vóxeles de la geometría, el trazado de conos y el uso de estructuras jerárquicas de datos y pre-filtrado.

Este trabajo se desarrolla en el contexto de un ambiente académico.
La implementación es open source y apunta a ser un recurso didáctico útil para otras personas intentando aprender sobre distintas técnicas de iluminación.

\section{Motivación}

% Por qué queríamos hacer algo con vóxeles?

Este trabajo surge del interés de los miembros del equipo en técnicas de iluminación global y en vóxeles como primitiva de renderizado.
Luego de dos cursos de computación gráfica en los que se trata por un lado la creación de ambientes interactivos y por otro la generación de imágenes realistas usando iluminación global, se buscó un algoritmo que permitiera ambas, iluminación global en tiempo real.

\section{Organización del documento}

Las siguientes secciones de este informe se organizan de la siguiente manera.
El capítulo 2 se enfoca en analizar trabajos anteriores y proporcionar el trasfondo necesario para entender vóxel cone tracing.
El capítulo 3 se enfoca en detallar cómo funciona el algoritmo y la estructura de datos utilizada para implementarlo.
El capítulo 4 presenta las decisiones tomadas respecto al desarrollo y proporciona pseudocódigo para entender el algoritmo en mayor detalle.
El capítulo 5 muestra los resultados de los experimentos realizados con la herramienta implementada que se presentan en forma de gráficas, tablas e imágenes.
Finalmente, el capítulo 6 resume los resultados y conclusiones de este trabajo y presenta las features y arreglos que se podrían implementar a futuro y por qué no entraron en el alcance de este proyecto.
