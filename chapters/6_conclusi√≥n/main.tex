\graphicspath{{chapters/6_conclusión/figures/}}

% En este capítulo se evalúan los resultados alcanzados y
% dificultades encontradas, se establece lo que se planteó hacer y lo que se hizo
% realmente, cuales fueron los aportes, se muestran posibles extensiones al trabajo, se
% realiza una autocrítica de lo que se hizo y lo que faltó (por problemas de tiempo,
% recursos, cómo se puede continuar, qué cosas hacer, prioridades, etc.) y se incluye
% información sobre la gestión del proyecto, si aplica

%%%%%%%%%%%%%%%%%%%%%%%%%%%%%%%%%%%%%%%%%%%%%%%%%%%%%%%%%%%%%%%%%%%%%%%%%%%%%%%%%%%%%%%%%%%%%%%%
% Parte final
%%%%%%%%%%%%%%%%%%%%%%%%%%%%%%%%%%%%%%%%%%%%%%%%%%%%%%%%%%%%%%%%%%%%%%%%%%%%%%%%%%%%%%%%%%%%%%%%

\chapter{Conclusiones y Trabajo Futuro}\label{chap:conclusions}

En este proyecto se logró una implementación práctica y funcional de la información provista por un artículo de revista \cite{voxel-cone-tracing} y una tesis \cite{gigavoxels}.
El objetivo principal fue estudiar y comprender a profundidad el algoritmo de Voxel Cone Tracing, enfrentando los retos de llevarlo a un entorno funcional y documentando el proceso de manera que sirva como recurso útil para otros interesados en este tema.
Así, se cumplieron los objetivos principales planteados al inicio del trabajo.

\section{Principales conclusions}

Se consiguió implementar el algoritmo de Voxel Cone Tracing.
Con el mismo se logró generar un rendering, en el hardware intermedio probado y en una escena compleja, con valores de cuadros por segundo cercanos al tiempo real.

A lo largo del proyecto, se profundizó en el entendimiento del algoritmo y en los desafíos del desarrollo en GPU.
Se dejó un recurso, detallando el algoritmo paso a paso, y el codigo queda abierto como ejemplo para futuros desarrolladores.

Este proyecto no solo permitió desarrollar un código concreto, sino que también sirvió como ejercicio para materializar conceptos teóricos en aplicaciones reales.


El desarrollo del proyecto presentó diversos desafíos.
Programar en la GPU implicó un cambio de paradigma respecto al modelo tradicional en CPU.
Las restricciones arquitectónicas y el diseño del pipeline de cómputo representaron una curva de aprendizaje significativa.
La depuración en GPU, limitada por la falta de herramientas tradicionales, requirió el desarrollo de utilidades personalizadas para visualizar los estados intermedios del algoritmo y diagnosticar problemas.

\section{Trabajo futuro}

El proyecto presenta varias oportunidades para mejorar y ampliar la implementación actual:

\begin{itemize}
    \item Soporte para objetos dinámicos: Implementar el manejo de objetos dinámicos en la escena y actualizar el octree en tiempo real, una característica considerada pero no incluida en este trabajo.
    \item Migración a Vulkan: Portar la implementación a Vulkan para aprovechar un control más detallado de la GPU, potencialmente mejorando el rendimiento y la flexibilidad.
    \item Mejora de herramientas de visualización: Desarrollar herramientas más avanzadas para inspeccionar los estados intermedios del algoritmo, facilitando la depuración y la investigación.
\end{itemize}

Los resultados y experiencias obtenidos proporcionan una base sólida para futuros desarrollos de Voxel Cone Tracing y otras técnicas de iluminación global.