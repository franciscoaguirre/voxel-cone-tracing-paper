\graphicspath{{chapters/6_conclusión/figures/}}

% En este capítulo se evalúan los resultados alcanzados y
% dificultades encontradas, se establece lo que se planteó hacer y lo que se hizo
% realmente, cuales fueron los aportes, se muestran posibles extensiones al trabajo, se
% realiza una autocrítica de lo que se hizo y lo que faltó (por problemas de tiempo,
% recursos, cómo se puede continuar, qué cosas hacer, prioridades, etc.) y se incluye
% información sobre la gestión del proyecto, si aplica

%%%%%%%%%%%%%%%%%%%%%%%%%%%%%%%%%%%%%%%%%%%%%%%%%%%%%%%%%%%%%%%%%%%%%%%%%%%%%%%%%%%%%%%%%%%%%%%%
% Parte final
%%%%%%%%%%%%%%%%%%%%%%%%%%%%%%%%%%%%%%%%%%%%%%%%%%%%%%%%%%%%%%%%%%%%%%%%%%%%%%%%%%%%%%%%%%%%%%%%

\chapter{Conclusiones y Trabajo Futuro}\label{chap:conclusions}

Este proyecto logró convertir un artículo académico en una implementación práctica y funcional de código abierto.
El objetivo principal fue estudiar y comprender a profundidad el algoritmo de Voxel Cone Tracing, enfrentando los retos de llevarlo a un entorno funcional y documentando el proceso de manera que sirva como recurso útil para otros interesados en este tema.
Así, se cumplieron los objetivos principales planteados al inicio del trabajo.

\section{Logros destacados}

Se consiguió implementar el algoritmo de manera que, en el hardware intermedio probado y en una escena compleja, se lograron valores de cuadros por segundo cercanos al tiempo real.
Los resultados de rendimiento obtenidos fueron de 10, 37 y 60 cuadros por segundo para la mejor, intermedia y peor aproximación respectivamente.

A lo largo del proyecto, se profundizó en el entendimiento del algoritmo y en los desafíos del desarrollo en GPU.
Se dejó un buen recurso, detallando el algoritmo paso a paso, y un ejemplo para futuros desarrolladores.

Más allá del programa implementado, fue un buen ejemplo de convertir un artículo académico en una implementación concreta.
Fue muy valioso cerrar la brecha entre teoría y práctica.

\section{Retos enfrentados}

El desarrollo del proyecto tuvo sus retos.

Programar en la GPU implicó un cambio de paradigma respecto al modelo tradicional en CPU.
Las restricciones de la arquitectura y el diseño del ducto de cómputo supusieron una curva de aprendizaje significativa.

La depuración en GPU, limitada por la falta de herramientas tradicionales, obligó al desarrollo de utilidades personalizadas para visualizar los estados intermedios del algoritmo y diagnosticar problemas.

\section{Propuestas para trabajo futuro}

El proyecto abre múltiples lineas de trabajo que podrían extender y mejorar la implementación actual.

\begin{itemize}
    \item Soporte para objetos dinámicos\\
        Incorporar la capacidad de manejar objetos dinámicos en la escena, adaptando el octree en tiempo real.
        Aunque esta posibilidad fue considerada durante el diseño, quedó fuera del alcance del proyecto.
    \item Migración a Vulkan\\
        Explorar la implementación en Vulkan permitiría experimentar con el control más granular que ofrece sobre la GPU, evaluando mejoras potenciales en rendimiento y flexibilidad.
    \item Mejores herramientas interactivas de visualización\\
        Desarrollar herramientas interactivas más robustas para inspeccionar los estados intermedios del algoritmo, facilitando tanto la depuración como la exploración académica.
\end{itemize}

Los resultados obtenidos y las experiencias documentadas constituyen una base sólida para inspirar y facilitar futuros desarrollos de \textit{Voxel Cone Tracing} o de otras técnicas de iluminación global.
