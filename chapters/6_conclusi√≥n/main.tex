\graphicspath{{chapters/6_conclusión/figures/}}

% En este capítulo se evalúan los resultados alcanzados y
% dificultades encontradas, se establece lo que se planteó hacer y lo que se hizo
% realmente, cuales fueron los aportes, se muestran posibles extensiones al trabajo, se
% realiza una autocrítica de lo que se hizo y lo que faltó (por problemas de tiempo,
% recursos, cómo se puede continuar, qué cosas hacer, prioridades, etc.) y se incluye
% información sobre la gestión del proyecto, si aplica

%%%%%%%%%%%%%%%%%%%%%%%%%%%%%%%%%%%%%%%%%%%%%%%%%%%%%%%%%%%%%%%%%%%%%%%%%%%%%%%%%%%%%%%%%%%%%%%%
% Parte final
%%%%%%%%%%%%%%%%%%%%%%%%%%%%%%%%%%%%%%%%%%%%%%%%%%%%%%%%%%%%%%%%%%%%%%%%%%%%%%%%%%%%%%%%%%%%%%%%

\chapter{Conclusiones y Trabajo Futuro}\label{chap:conclusions}

Se logró pasar de un artículo académico a una implementación práctica, open source.
Se enfocó en aprender a fondo el algoritmo y en documentar la implementación para que sea un recurso útil para otros estudiantes interesados en este tema.
Dicho esto, se cumplieron los objetivos principales.

La mayoría del tiempo fue dedicado a entender los conceptos y la programación en GPU mediante el uso de \textit{compute shaders}.
Esta fue la principal dificultad encontrada en el transcurso del trabajo, implementar un algoritmo principalmente en la GPU.
El equipo tiene mayoritariamente experiencia programando en la CPU y manipulando etapas del pipeline gráfico, pero el uso extensivo de \textit{compute shaders}, texturas (lineales, 2D, 3D), imágenes y \textit{samplers} fue algo que sin dudas enlenteció el desarrollo.

A su vez, en el transcurso del trabajo, la falta de planificación, fijación de objetivos y fechas límite fue otro problema.

Se plantean posibles tareas de un trabajo futuro:
\begin{itemize}
    \item Objetos dinámicos\\
        Es posible tener objetos dinámicos en la escena, que al moverse subdividen el \textit{octree} nuevamente.
        La estructura fue implementada con esto en mente, pero quedó fuera del alcance.
    \item Pasar a Vulkan\\
        Pasar el programa a Vulkan sería un buen experimento para aprender y medir la mejora en eficiencia que resulta, dado que Vulkan permite mejor control de la GPU lo que ofrece más oportunidades de optimización.
    \item Mejores herramientas interactivas de exploración\\
        Mejorar las herramientas interactivas para la exploración del algoritmo.
\end{itemize}
