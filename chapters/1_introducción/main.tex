\chapter{Introducción}

% Aquí se motiva el trabajo, se plantea y define el problema, se deja claro cuales son los objetivos (general del proyecto, si correspondiese o si está inmerso en un proyecto de mayor alcance, y los específicos), se plantean los resultados esperados, se establecen resumidamente las conclusiones y se describe la
% organización general del documento.

El problema de la iluminación global en tiempo real ha sido muy estudiado.
Resolverlo es uno de los objetivos largamente buscados de la computación gráfica, debido a que tiene una alta importancia en varias industrias, como la de los videojuegos, la del cine, las simulaciones físicas, entre otros.
Existen varios métodos para resolver el problema de la iluminación global.
La mayoría de ellas se basan en el trazado de rayos.
Esto consiste en trazar rayos a partir de la cámara hacia la escena y simular los caminos que recorren los fotones.
Esto no funciona en tiempo real.
Se han hecho varias optimizaciones a lo largo de los años para lograrlo, como por ejemplo simplificaciones en la geometría, el uso de estructuras jerárquicas, clustering, entre otras.
Más allá de los avances, el problema continúa siendo un área activa de investigación.
Métodos recientes utilizan hardware especializado para lograr alcanzar tiempos interactivos.
Una técnica reciente (2019) es DLSS (\textit{Deep Learning Super Sampling}), que consiste en ejecutar estos algoritmos en resoluciones de pantalla mucho menores a la objetivo, y luego, utilizar aprendizaje automático para agrandar esa pantalla a la objetivo.

Este trabajo se centra en \textit{voxel cone tracing}, un algoritmo de iluminación global que funciona en tiempo real sin necesidad de hardware específico.
El objetivo del trabajo es aprender sobre este algoritmo, crear una implementación del mismo, y probar su eficiencia en hardware moderno.
Este algoritmo fue propuesto por Crassin et al en 2011 \cite{voxel-cone-tracing}, cuando no existían muchas soluciones para el problema de iluminación global en tiempo real, y la demanda estaba creciendo debido a la importancia de la industria de los videojuegos.
Se basa en una representación de vóxeles de la geometría, el trazado de conos y el uso de estructuras jerárquicas de datos y pre-filtrado para reducir los cálculos necesarios y así alcanzar tiempos interactivos.
No sufre de problemas de ruido y provee una buena calidad de imágenes con un rendimiento casi independiente de la complejidad de la escena.
Computa hasta dos rebotes de la luz en su camino desde el emisor hacia la cámara, lo que permite incorporar el componente principal de la luz indirecta.

Este trabajo surge del interés de los miembros del equipo en técnicas de iluminación global y en vóxeles como primitiva de renderizado.
Luego de dos cursos de computación gráfica en los que se trata por un lado la creación de ambientes interactivos y por otro la generación de imágenes realistas usando iluminación global, se buscó un algoritmo que permitiera ambas, iluminación global en tiempo real.
El trabajo se realiza en el contexto de un ambiente académico, la implementación es open source y apunta a ser un recurso didáctico útil para otras personas intentando aprender sobre distintas técnicas de iluminación.

Se esperaba lograr una implementación funcional del algoritmo y se logró.

\section{Organización del documento}

Las siguientes secciones de este informe se organizan de la siguiente manera.
El capítulo 2 se enfoca en analizar trabajos anteriores y proporcionar el trasfondo necesario para entender vóxel cone tracing.
El capítulo 3 se enfoca en detallar cómo funciona el algoritmo y la estructura de datos utilizada para implementarlo.
El capítulo 4 presenta algunas decisiones tomadas respecto al desarrollo.
El capítulo 5 muestra los resultados de los experimentos realizados con la aplicación implementada que se presentan en forma de tablas e imágenes.
Finalmente, el capítulo 6 resume los resultados y conclusiones de este trabajo y presenta las funcionalidades y arreglos para implementar a futuro.
