\documentclass{prgrado}

\usepackage[
    backend=biber,
    style=alphabetic,
]{biblatex}

% Hace que el + de las citas cuando hay demasiados autores sea un superscript
\renewcommand*{\labelalphaothers}{\textsuperscript{+}}

% Agranda las filas de las tablas
\renewcommand{\arraystretch}{1.5}

% Traté de que dijera "Tabla" en las captions de las tablas en lugar de "Cuadro"
% No funcionó :(
% \usepackage{babel}[spanish]
% \usepackage{caption}
% \renewcommand{\tablename}{Tabla}

\usepackage{subcaption}
\usepackage{amsfonts}
\usepackage{amsmath}

\addbibresource{referencias.bib}

% título del proyecto (debe escribirse en minúscula, con excepción de la
% letra inicial de la primera palabra y nombres propios)
\title{Fingxels}

% autores
\author{Francisco Aguirre, Felipe Pizzorno} % TODO: Ver si podemos poner un subtítulo

% fecha de la defensa
\date{\today} % TODO: Change

% supervisor
\supervisor{Eduardo Fernández}

% cosupervisor (comentar si no tiene)
\cosupervisor{Jose Pedro Aguerre}

% licencia creative commons del documento
% opciones: by, by-sa, by-nd, by-nc, by-nc-sa y by-nc-nd
% opción por defecto: by-nc-nd versión 4.0
\cclicense{by}{4.0}

%%%%%%%%%%%%%%%%%%%%%%%%%%%%%%%%%%%%%%%%%%%%%%%%%%%%%%%%%%%%%%%%%%%%%%%%%%%%%%%%%%%%%%%%%%%%%%%%

%links con colores 
\hypersetup{
  colorlinks   = true,
  urlcolor     = blue,
  linkcolor    = blue,
  citecolor    = blue
}

\graphicspath{{figures/}}

\begin{document}

%%%%%%%%%%%%%%%%%%%%%%%%%%%%%%%%%%%%%%%%%%%%%%%%%%%%%%%%%%%%%%%%%%%%%%%%%%%%%%%%%%%%%%%%%%%%%%%%
% Parte inicial
%%%%%%%%%%%%%%%%%%%%%%%%%%%%%%%%%%%%%%%%%%%%%%%%%%%%%%%%%%%%%%%%%%%%%%%%%%%%%%%%%%%%%%%%%%%%%%%%

\frontmatter % numeración en romanos, capitulos sin numerar

% carátula
\maketitle

%%%%%%%%%%%%%%%%%%%%%%%%%%%%%%%%%%%%%%%%%%%%%%%%%%%%%%%%%%%%%%%%%%%%%%%%%%%%%%%%%%%%%%%%%%%%%%%%

% agradecimientos
\chapter*{Agradecimientos}

Agradecer, siempre es bueno agradecer. % TODO: Hacer a lo último

%%%%%%%%%%%%%%%%%%%%%%%%%%%%%%%%%%%%%%%%%%%%%%%%%%%%%%%%%%%%%%%%%%%%%%%%%%%%%%%%%%%%%%%%%%%%%%%%

% resumen
\chapter*{Resumen}

% El resumen (200-500 palabras) debe dar una idea completa de todo el
% proyecto, mencionando claramente los formalismos, técnicas, herramientas y lenguajes
% utilizados. No debe limitarse a describir el problema abordado, sino que debe describir
% la solución del problema, con una evaluación de la misma. No debe incluir referencias
% bibliográficas ni referencias a otras partes del informe. Tampoco debe utilizar
% acrónimos sin explicar su significado.

En este trabajo se aborda el problema de la iluminación global en tiempo real, un problema desafiante y relevante en la industria de los videojuegos, cine, simulaciones y la computación gráfica en general.
La mayoría de las técnicas de iluminación global se basan en el método de Monte Carlo, que las hace computacionalmente costosas, requiriendo hardware especializado y técnicas de aprendizaje automático para alcanzar el tiempo real.

En este contexto, se hace foco en el algoritmo de \textit{voxel cone tracing}, una técnica de trazado de conos que usa una estructura de datos basada en vóxeles que se ha demostrado prometedora por su capacidad para producir efectos de iluminación global de alta calidad en tiempo real sin necesidad de hardware especializado.

El objetivo principal de este trabajo es crear una implementación open source del algoritmo de \textit{voxel cone tracing} y probarla en hardware moderno. Esta implementación no solo demuestra la viabilidad y eficacia de la técnica en un entorno de hardware actual, sino que también es un recurso didáctico valioso para cualquier persona interesada en aprender acerca de computación gráfica, iluminación global y su implementación.

Para esta implementación, se utilizaron principalmente las herramientas Rust y OpenGL, debido a su alto rendimiento, abundancia de documentación y simplicidad.
El algoritmo corre exclusivamente en la GPU, aprovechando la alta paralelización que esta provee.
Su desarrollo se realizó en Linux, pero el programa no está limitado a este sistema operativo dado que puede ser portado fácilmente a otros.

Para evaluar la implementación, se realizaron experimentos en hardware moderno y se compararon los resultados con implementaciones previas de la técnica.

El código del motor se encuentra en el siguiente repositorio:

\url{https://github.com/franciscoaguirreperez/voxel-cone-tracing}

\hfill \break
\keywords{Voxel cone tracing, Iluminación global, OpenGL, Rust}

%%%%%%%%%%%%%%%%%%%%%%%%%%%%%%%%%%%%%%%%%%%%%%%%%%%%%%%%%%%%%%%%%%%%%%%%%%%%%%%%%%%%%%%%%%%%%%%%

% índice
\tableofcontents
\newpage


%%%%%%%%%%%%%%%%%%%%%%%%%%%%%%%%%%%%%%%%%%%%%%%%%%%%%%%%%%%%%%%%%%%%%%%%%%%%%%%%%%%%%%%%%%%%%%%%
% Parte central
%%%%%%%%%%%%%%%%%%%%%%%%%%%%%%%%%%%%%%%%%%%%%%%%%%%%%%%%%%%%%%%%%%%%%%%%%%%%%%%%%%%%%%%%%%%%%%%%

\mainmatter %numeración en arábicos, numerar capítulos 

% introducción
\chapter{Introducción}

% Aquí se motiva el trabajo, se plantea y define el problema, se deja claro cuales son los objetivos (general del proyecto, si correspondiese o si está inmerso en un proyecto de mayor alcance, y los específicos), se plantean los resultados esperados, se establecen resumidamente las conclusiones y se describe la
% organización general del documento.

El problema de la iluminación global en tiempo real ha sido muy estudiado.
Resolverlo es uno de los objetivos largamente buscados de la computación gráfica, debido a que tiene una alta importancia en varias industrias, como la de los videojuegos, la del cine, las simulaciones físicas, entre otros.
Existen varios métodos para resolver el problema de la iluminación global.
La mayoría de ellas se basan en el trazado de rayos.
Esto consiste en trazar rayos a partir de la cámara hacia la escena y simular los caminos que recorren los fotones.
Esto no funciona en tiempo real.
Se han hecho varias optimizaciones a lo largo de los años para lograrlo, como por ejemplo simplificaciones en la geometría, el uso de estructuras jerárquicas, clustering, entre otras.
Más allá de los avances, el problema continúa siendo un área activa de investigación.
Métodos recientes utilizan hardware especializado para lograr alcanzar tiempos interactivos.
Una técnica reciente (2019) es DLSS (\textit{Deep Learning Super Sampling}), que consiste en ejecutar estos algoritmos en resoluciones de pantalla mucho menores a la objetivo, y luego, utilizar aprendizaje automático para agrandar esa pantalla a la objetivo.

Este trabajo se centra en \textit{voxel cone tracing}, un algoritmo de iluminación global que funciona en tiempo real sin necesidad de hardware específico.
El objetivo del trabajo es aprender sobre este algoritmo, crear una implementación del mismo, y probar su eficiencia en hardware moderno.
Este algoritmo fue propuesto por Crassin et al en 2011 \cite{voxel-cone-tracing}, cuando no existían muchas soluciones para el problema de iluminación global en tiempo real, y la demanda estaba creciendo debido a la importancia de la industria de los videojuegos.
Se basa en una representación de vóxeles de la geometría, el trazado de conos y el uso de estructuras jerárquicas de datos y pre-filtrado para reducir los cálculos necesarios y así alcanzar tiempos interactivos.
No sufre de problemas de ruido y provee una buena calidad de imágenes con un rendimiento casi independiente de la complejidad de la escena.
Computa hasta dos rebotes de la luz en su camino desde el emisor hacia la cámara, lo que permite incorporar el componente principal de la luz indirecta.

Este trabajo surge del interés de los miembros del equipo en técnicas de iluminación global y en vóxeles como primitiva de renderizado.
Luego de dos cursos de computación gráfica en los que se trata por un lado la creación de ambientes interactivos y por otro la generación de imágenes realistas usando iluminación global, se buscó un algoritmo que permitiera ambas, iluminación global en tiempo real.
El trabajo se realiza en el contexto de un ambiente académico, la implementación es open source y apunta a ser un recurso didáctico útil para otras personas intentando aprender sobre distintas técnicas de iluminación.

Se esperaba lograr una implementación funcional del algoritmo y se logró.

\section{Organización del documento}

Las siguientes secciones de este informe se organizan de la siguiente manera.
El capítulo 2 se enfoca en analizar trabajos anteriores y proporcionar el trasfondo necesario para entender vóxel cone tracing.
El capítulo 3 se enfoca en detallar cómo funciona el algoritmo y la estructura de datos utilizada para implementarlo.
El capítulo 4 presenta algunas decisiones tomadas respecto al desarrollo.
El capítulo 5 muestra los resultados de los experimentos realizados con la aplicación implementada que se presentan en forma de tablas e imágenes.
Finalmente, el capítulo 6 resume los resultados y conclusiones de este trabajo y presenta las funcionalidades y arreglos para implementar a futuro.


% antecedentes
\graphicspath{{chapters/2_antecedentes/figures/}}

\chapter{Revisión de antecedentes}

% Idea:
% 1. ¿Qué es la luz?
% 2. Unidades de luz
% 3. Iluminación local y global
% 4. BRDF
% 5. Trazado de rayos
% 6. Trazado de conos
% 7. Photon Mapping
% 8. Vóxeles (acá van a haber integrales, rendering científico)
% 9. Octrees
% 10. Ducto gráfico
% 11. Rendering diferido
% ... no se si esto va acá ...
% 12. Teoría de pre-filtrado
% 13. Ray marching

En este capítulo se explicaran en detalle los conceptos teóricos y los antecedentes más importantes para entender el algoritmo de \textit{voxel cone tracing} y la implementación desarrollada.

\section{¿Qué es la luz?}

Antes de definir el problema de la iluminación global en computación gráfica y hablar de diversos trabajos que se han realizado al respecto, conviene dar un paso atrás y preguntarnos, ¿qué es la luz?
Esta pregunta es relevante porque, más allá de las variaciones, la gran mayoría de técnicas de iluminación global se basan en modelar a la luz físicamente.

El libro \textit{Physically Based Rendering}, de Pharr, Jakob y Humphreys \cite{pbr} brinda una explicación clara y resumida de la historia de la interacción entre los humanos y la luz.
La percepción a partir de la luz es central para nuestra existencia.
La incógnita de la naturaleza de la luz ha ocupado las mentes de grandes filósofos y físicos desde el comienzo de los tiempos.
La antigua escuela filosófica hinduista de Vaisheshika (siglos 5 a 6 antes de cristo) veía a la luz como una colección de pequeñas partículas viajando a través de rayos a una alta velocidad.
En el siglo 5 antes de cristo, el filósofo griego Empédocles postulaba que un fuego divino emergía de los ojos humanos y combinado con los rayos de luz del sol producía visión.
Entre los siglos 18 y 19, eruditos como Isaac Newton, Thomas Young, Augustin-Jean Fresnel desarrollaron teorías conflictivas, donde algunas modelaban la luz como consecuencia de la propagación de ondas, y otras de partículas.
Al mismo tiempo, André-Marie Ampère, Joseph-Louis Lagrange, Carl Friedrich Gauß, y Michael Faraday investigaban las relaciones entre electricidad y magnetismo que culminaron en la repentina y dramática unificación de James Clerk Maxwell en una teoría combinada conocida como \textbf{electromagnetismo}, \cite{maxwell-equations}.

La luz es una manifestación con propiedades de onda.
El movimiento de partículas eléctricamente cargadas, como electrones dentro del filamento de una bombilla, produce un disturbio en un campo eléctrico circundante que se propaga hacia fuera de la fuente.
La oscilación eléctrica también produce una oscilación secundaria de un campo magnético, que a su vez refuerza la oscilación del campo eléctrico, y así sucesivamente.
La interacción entre estos dos campos da lugar a una onda que se autopropaga y puede viajar distancias extremadamente largas.
Una representación de esto se puede ver en la figura \ref{fig:electromagnetic-wave}, en la que el campo eléctrico (azul) y el magnético (rojo) son perpendiculares el uno al otro y se propagan, avanzando a lo largo de un eje central.

\begin{figure}
    \centering
    \includegraphics[width=\textwidth]{electromagnetic_wave.png}
    \caption{Representación de una onda electromagnetica propagandose por el espacio}
    \label{fig:electromagnetic-wave}
\end{figure}

A principios del siglo XX, los trabajos liderados por Max Planck, Max Born, Erwin Schrödinger, y Werner Heisenberg condujeron a otro cambio sustancial en el entendimiento de la luz.
A nivel microscópico, las propiedades elementales como energía y momento solo pueden existir como un múltiplo entero de una cantidad base conocida como un \textbf{cuanto}.
En el caso de oscilaciones electromagnéticas, este cuanto se conoce como \textbf{fotón}.
La luz existe tanto como onda y como partícula \cite{quantum-light}.

Afortunadamente, el complejo comportamiento de onda de la luz aparece en escalas muy pequeñas, por lo que, para la computación gráfica, en la mayoría de los casos, se la puede tratar como partícula.
Esto simplifica los cálculos \cite{pbr}.

Tratar a la luz como una partícula es el campo de la óptica geométrica, en contraposición a la óptica física.
La óptica geométrica trata a la luz como rayos que se mueven en líneas rectas.
La óptica física aborda la luz desde el punto de vista de su naturaleza ondulatoria, centrándose en fenómenos como la interferencia, la difracción, la polarización.
Si asumimos que las irregularidades de las superficies son en general mucho más grandes que la longitud de onda de la luz, estos efectos no ocurren, y se puede tratar la luz como rayos \cite{rtr}.
Esto remueve otros fenómenos como la fluorescencia, la fosforescencia y la polarización.

% Esto asume varias otras cosas, como linealidad, conservación de la energía, falta de polarización, fluorescencia y fosforescencia, entre otros.

\section{Radiometría: Unidades de la luz}\label{sec:radiometry}

La radiometría provee una serie de herramientas para describir la propagación de la luz.
Para simular luz, es necesario un manejo de las unidades básicas involucradas.
Algunas de estas son la energía radiante, el flujo radiante, la radiosidad, la irradiancia, la intensidad radiante, y la radiancia.
La descripción de estas unidades surgen de \cite{rtr} y \cite{pbr}.

La \textbf{energía radiante} es la energía de la radiación electromagnética de la luz.
Se mide en joules y se denota con el símbolo $Q$.
Las fuentes de iluminación emiten fotones, y cada uno posee una longitud de onda y una energía particular.
Un fotón con longitud de onda $\lambda$ tiene una energía $Q = \frac{hc}{\lambda}$, donde $c$ es la velocidad de la luz y $h$ es la constante de Planck, el cuanto.

El \textbf{flujo radiante}, o potencia, es la energía que pasa por una superficie o región del espacio por unidad de tiempo: $\Theta = \frac{dQ}{dt}$.
Se mide en joules por segundo, o watts.
En la figura \ref{fig:point-light-flux} se representa en 2D una luz puntual emitiendo fotones en todas las direcciones.
Los círculos de la figura son áreas en las que se mide el flujo radiante que pasa por ellas.
Por cada círculo pasa el mismo flujo radiante, dado que la energía y el tiempo son los mismos.

\begin{figure}[ht]
    \centering
    \includegraphics[width=0.25\textwidth]{point-light-flux.png}
    \caption{Flujo radiante en una luz puntual}
    \label{fig:point-light-flux}
\end{figure}

Es útil también considerar el área por la cual pasa un flujo radiante.
Podemos definir esto como $E = \frac{\Theta}{A}$.
Esta cantidad se llama o \textbf{radiosidad} o \textbf{irradiancia} dependiendo de si el flujo está llegando o saliendo de una superficie.
Estas medidas tienen unidad watts por metro cuadrado.
En la figura \ref{fig:point-light-flux}, la irradiancia en el círculo externo es menor que en el interno, dado que el área aumenta cuadráticamente con la distancia.

Para definir la próxima unidad, es necesario definir el \textbf{ángulo sólido}, que es la extensión a 3D del ángulo bidimensional.
En 2D, un ángulo mide el tamaño de un conjunto continuo de direcciones en un plano.
Para medir esto, se mide el largo del arco resultante de la intersección de este conjunto con un círculo de radio 1.
El largo de este arco se mide en radianes.
De igual manera, un ángulo sólido mide el tamaño de un conjunto de direcciones en el espacio.
Para lograr esto, se mide el área de la intersección del conjunto de direcciones con una esfera de radio 1.
La unidad de medida es el estereorradián.
El ángulo sólido se representa con el símbolo $\omega$.
En la figura \ref{fig:steradian} se puede ver un cono con un ángulo sólido de 1 estereorradián.
El ángulo sólido que abarca todas las direcciones posibles mide $4\pi$ estereorradianes.

\begin{figure}[ht]
    \centering
    \includegraphics[width=0.4\textwidth]{steradians.png}
    \caption{Un cono con un ángulo sólido de 1 estereorradián removido de una esfera. Fuente: \cite{rtr}}
    \label{fig:steradian}
\end{figure}

La \textbf{intensidad radiante} es el flujo radiante dada una dirección, o mejor dicho, un ángulo sólido.
Se denota $I(\omega) = \frac{d\Theta}{d\omega}$ y se mide en watts por estereorradián.

Llegamos a la unidad radiométrica más importante, la \textbf{radiancia}.
La radiancia es la cantidad de flujo radiante emitida, reflejada, transmitida o recibida por una superficie por unidad de área y por unidad de ángulo sólido.
Se denota como $L$ y se mide en watts por metro cuadrado por estereorradián.

$$
L = \frac{d^2\Theta}{(dA\,\cos\theta)d\omega}
$$

La radiancia es lo que miden los sensores, como los ojos o cámaras.
El objetivo de evaluar una ecuación de sombreado es calcular la radiancia a lo largo de un rayo, desde el punto de vista de la cámara.

Lo anteriormente expuesto son algunas unidades de la radiometria que trata unicamente con cantidades fisicas de la luz.
En contraposicion, un campo relacionado, la fotometría mide la luz en funcion de la percepcion del ojo humano.

La radiometría trata únicamente con cantidades físicas de la luz.
En contraposición, un campo relacionado, la fotometría, mide la luz tal como es percibida por el ojo humano, teniendo en cuenta la sensibilidad de este a distintas longitudes de onda.
En la tabla \ref{table:light_units} se muestran algunas unidades equivalentes en radiometría y fotometría.

\begin{table}
\centering
\begin{tabular}{|c|c|}
    \hline
    \textbf{Radiometría (unidad)} & \textbf{Fotometría (unidad)} \\
    \hline
    Flujo radiante ($W$) & Flujo luminoso ($lumen, lm$) \\
    \hline
    Radiosidad $\left(\frac{W}{m^2}\right)$ & Emitancia luminosa $\left(\frac{lm}{m^2} = lux, lx\right)$ \\
    \hline
    Irradiancia $\left(\frac{W}{m^2}\right)$ & Iluminancia ($lx$) \\
    \hline
    Intensidad radiante $\left(\frac{W}{sr}\right)$ & Intensidad luminosa ($candela, cd$) \\
    \hline
    Radiancia $\left(\frac{W}{(m^2sr)}\right)$ & Luminancia $\left(\frac{cd}{m^2} = nit\right)$ \\
    \hline
\end{tabular}
\caption{Unidades de radiometría y fotogrametría}
\label{table:light_units}
\end{table}

\section{Iluminación local e iluminación global}\label{sec:local_vs_global}

En computación gráfica, la iluminación es crucial para agregar realismo a una escena, sin embargo, es también uno de los aspectos más desafiantes desde el punto de vista computacional, debido a la complejidad de simular cómo la luz interactúa con los objetos y el entorno.

\subsection{Iluminación local}

La iluminación local es un enfoque más simple y menos demandante computacionalmente, que ayuda a dar una sensación de tridimensionalidad.
Consiste en calcular la iluminación directa para cada objeto individualmente, sin considerar la iluminación indirecta que existe entre objetos vecinos.
Al no calcular la interacción de la luz con otros objetos, se simplifican significativamente los cálculos necesarios para generar la imagen.

Sus ventajas son su simplicidad y su eficiencia.
Su mayor desventaja es que no abarca fenómenos como la refracción, las cáusticas y el sangrado, entre otros.

Un modelo ampliamente utilizado para iluminación local es el de Blinn-Phong, \cite{blinn-phong}.
Este posee tres componentes principales: luz \textbf{ambiente}, \textbf{difusa} y \textbf{especular}.
La luz ambiente es uniforme y está presente en toda la escena.
No proviene de una fuente de luz en particular y simula el efecto de la luz indirecta que proviene del resto de la escena.
La luz difusa representa el efecto de la luz directa que incide sobre una superficie rugosa y se refleja en todas las direcciones.
Depende del ángulo entre la dirección de la luz y la normal de la superficie.
La luz especular simula el brillo que se ve cuando la luz se refleja sobre superficies pulidas.
Este componente depende de la dirección de vista, de la normal de la superficie y de la dirección de la fuente luminosa en cada punto de la superficie.

% TODO: Fotos, ambiente, difusa, especular y todo junto

\begin{figure}
    \begin{center}

    \begin{subfigure}{.24\textwidth}
        \includegraphics[width=\textwidth]{sphere-unshaded.png}
        \caption{Sin sombreado}
    \end{subfigure}

    \begin{subfigure}{.24\textwidth}
        \includegraphics[width=\textwidth]{sphere-diffuse.png}
        \caption{Con sombreado difuso}
    \end{subfigure}

    \end{center}

    \caption{Esfera iluminada mediante el módelo local de Blinn-Phong}
    \label{fig:sphere-blinn-phong}
\end{figure}

\subsection{Iluminación global}

En contraposición a la iluminación local, la iluminación global calcula la interacción completa de la luz con los objetos de la escena.
Se calcula la interacción de la luz entre distintos objetos al reflejarse, refractarse y dispersarse en el entorno.

Su principal ventaja es la mejora en el realismo y coherencia de la escena, mientras que su principal desventaja es la exigencia computacional y complejidad al implementar y optimizar.
Esto causa que sea difícil alcanzar tiempos interactivos.

Entre las técnicas más utilizadas se encuentra el trazado de rayos, la radiosidad, el \textit{photon mapping}, y el \textit{path tracing}, que serán presentadas a continuación.

En 1986, James T. Kajiya presentó la \textbf{ecuación de renderizado}, formalizando varios métodos para el cálculo de iluminación global como aproximaciones a la solución de una misma ecuación, \cite{rendering-equation}:

\begin{equation}\label{eq:rendering-equation}
    I(x, x') = g(x, x') \cdot \left[\epsilon(x, x') + \int_S f(x, x', x'') \cdot I(x', x'') \cdot dx'' \right]
\end{equation}

donde:
\begin{itemize}
    \item $I(x, x')$ se relaciona con al intensidad radiante que pasa del punto $x'$ al punto $x$.
    \item $g(x, x')$ es un término de "geometría", evalúa si hay algo interponiéndose en el camino de $x'$ a $x$.
    \item $\epsilon(x, x')$ se relaciona con la intensidad emitida desde $x'$ en dirección a $x$.
    \item $f(x, x', x'')$ es la función de distribución de reflectancia bidireccional (BRDF, por sus siglas en inglés). Se relaciona con la intensidad de luz reflejada desde $x''$ hacia $x$ a través de $x'$. En la sección \ref{sec:brdf} se analizan casos particulares de esta función.
    \item $S = \cup{S_i}$, el dominio de la integral, es la unión de todas las superficies de la escena. Esto significa que los puntos $x, x', x''$ varían a lo largo de todas las superficies.
\end{itemize}

En palabras, la ecuación \ref{eq:rendering-equation} establece que la intensidad que llega a un punto $x$ desde otro punto $x'$, es la intensidad que $x'$ emite en dirección a $x$ más la intensidad reflejada por $x'$ hacia $x$ desde cualquier otro punto de la escena.
Todo sujeto a si hay geometría en el camino de $x$ a $x'$, si es que $x'$ ``ve'' a $x$.

Los términos ``intensidad radiante'' e ``intensidad emitida'' tal y como son planteados no son exactamente ninguna de las unidades radiométricas vistas en la sección \ref{sec:radiometry} pero son cantidades que pueden derivarse de estas.

Esta ecuación es la base de varios métodos de iluminación global \cite{rtr}.
Se puede observar que esta ecuación tiene en cuenta toda la escena para calcular la luz en un punto.

La ecuación no presenta una solución analítica cerrada para la mayoría de los casos, por lo que se calculan soluciones numéricas para su resolución.

La ecuación \ref{eq:rendering-equation} considera toda la iluminación de la escena, incluyendo tanto luz directa como indirecta.
La \textbf{ecuación de reflectancia} se deriva de la de renderizado y se enfoca en el cálculo de la luz indirecta.

\begin{equation}\label{eq:reflectance-equation}
    L_o(p, \omega_o) = \int_{\Omega}{f(p, \omega_i, \omega_o) L_i(p, \omega_i) n \cdot \omega_i d\omega_i}
\end{equation}

Esta ecuación calcula la radiancia saliente $L_o$ de un punto $p$ y un pequeño ángulo sólido $\omega_o$ como una integral sobre el hemisferio $\Omega$ centrado en el punto.
Sus términos son muy parecidos a los de la ecuación de renderizado.
Vemos que ya no están el término geométrico, dado que se trata de direcciones y no puntos, ni el término de emisión, dado que solo se considera luz indirecta.
$f$ es la BRDF.
$L_i(p, \omega_i)$ es la radiancia entrante hacia $p$ por otra dirección $\omega_i$.
$n$ es la normal en el punto $p$ y $n \cdot \omega_i$ es el producto interno entre la normal y la dirección, que equivale al coseno del ángulo entre ellos.
La radiancia saliente es la suma por todo el hemisferio $\Omega$ centrado en $p$.
La suma de todas las radiancias a lo largo del hemisferio es la irradiancia, como se vió anteriormente.

% \begin{figure}
%     \centering
%     \includegraphics[width=\textwidth]{}
%     \caption{Hemisferio centrado en un punto $p$}
%     \label{fig:hemisphere}
% \end{figure}

\section{BRDF}\label{sec:brdf}

La \textbf{función de distribución de reflectancia bidireccional} (BRDF, por sus siglas en inglés), es una función $f(x, \omega, \omega')$ que dado un punto $x$, una dirección $\omega$ entrante y una saliente $\omega'$, retorna cuánta luz se refleja desde la dirección entrante hacia la saliente en el punto.

Está función se puede obtener usando tanto modelos analíticos como midiendo objetos reales con cámaras calibradas y fuentes de luz.
A lo largo de los años, se han propuesto varios BRDFs, tanto teóricos como empíricos \cite{review-of-brdf-models}.
Los BRDFs son propios de los materiales, y este es uniforme a lo largo de todo el material, por lo que no se diferencia entre dos puntos distintos $x$ y $x'$ del mismo material.

% TODO: Agregar una foto de la máquina utilizada para medir BRDFs
% TODO: Podemos hablar más de cómo se representan capaz

Dos casos teóricos e ideales son los siguientes:

\begin{itemize}
    \item Reflexión especular: $f(x, \omega, \omega') = \rho$ cuando $\omega$ es simétrico a $\omega'$ respecto a la normal. $0$ en otro caso.
    \item Reflexión difusa: $f(x, \omega, \omega') = \frac{\rho}{\pi}$ para todas las direcciones donde $\rho$ es la reflectividad de la superficie, es decir, la fracción de la energía reflejada con respecto a la energía incidente total.
\end{itemize}

La reflexión especular refleja un rayo solamente en un ángulo específico.
La reflexión difusa refleja la luz igual en todas las direcciones, con un mismo valor de BRDF para cada una de ellas.

Estos BRDFs son muy utilizados, dado que simplifican mucho los calculos y son fisicamente posibles, aunque no existan materiales reales con estas características de reflectancia.

\section{Trazado de rayos (\textit{ray tracing})}\label{sec:ray-tracing}

Uno de los métodos más antiguos y populares para calcular la iluminación global es el de trazado de rayos.
Este se basa en tratar a la luz como una partícula, y trazar el camino que toman los rayos de luz a través de la escena, calculando reflección, refracción y absorción del rayo cuando interseca con un objeto.
La popularidad de este método surge debido a su simplicidad y al gran realismo que agrega a las imágenes generadas.
Su primer uso como herramienta de computación gráfica para representar reflexión, refracción y sombras se atribuye a Turner Whitted, en 1980, \cite{whitted-1980}.

\textit{Ray tracing} lanza rayos desde la cámara a través de la grilla de píxeles hacia la pantalla.
Por cada rayo, se busca la intersección con el objeto más próximo.
El punto de intersección se prueba si está en sombra lanzando un nuevo rayo hacia todas las luces de la escena y comprobando si se intersecan con algún objeto.
Pueden surgir otros rayos desde la intersección.
Si la superficie es especular, se genera un rayo en la dirección simétrica a la dirección de vista.
Si la superficie es transparente, se genera un rayo en la dirección de refracción, gobernada por la ley de Snell\footnote{$n_1 \sin{\theta_1} = n_2 \sin{\theta_2}$}.
En la figura \ref{fig:whitted-ray-tracing} se puede ver una imagen generada por este método.

Este método sufre de \textit{aliasing}, los bordes irregulares.
Esto ocurre debido a una falta de resolución, es decir, cantidad de píxeles, que no logra capturar todo el detalle de la imagen.
Suele notarse al representar curvas.
Se puede ver en la figura \ref{fig:whitted-ray-tracing}.

% TODO: Agregar diagramas de 1) sombra, 2) reflexión y 3) refracción.

\begin{figure}
    \begin{center}
    \begin{subfigure}{.69\textwidth}
        \includegraphics[width=\textwidth]{whitted-ray-tracing.png}
        \caption{Escena entera}
    \end{subfigure}
    \begin{subfigure}{.3\textwidth}
        \includegraphics[width=\textwidth]{whitted-aliasing.png}
        \caption{\textit{Aliasing}}
    \end{subfigure}
    \caption{Escena renderizada con el trazado de rayos de Whitted. Implementación propia.}
    \label{fig:whitted-ray-tracing}
    \end{center}
\end{figure}

Kajiya propuso en 1986\cite{rendering-equation}, junto con la ecuación de renderizado, un método llamado \textit{path tracing}, una extensión de \textit{ray tracing} que utiliza integración de Monte Carlo para simular la dispersión de la luz.
En lugar de únicamente trazar los caminos de reflexión y refracción si corresponden, muestrea aleatoriamente todos los posibles caminos de la luz, esto incluye caminos en los que la luz se dispersa.
Es imposible muestrear todos los caminos posibles de la luz, por lo que un parámetro importante en este tipo de algoritmos es la cantidad de muestras que se toman.
Debido a la variancia del muestreo aleatorio, este método sufre de ruido, un granulado en la imagen final.
El mismo disminuye a medida que se toman más muestras.

% TODO: Imagen generada con path tracing y ruido

Tanto el \textit{aliasing} como el ruido pueden mitigarse con una técnica llamada \textit{supersampling}.
Consiste en tomar más de una muestra por píxel, lo que significa lanzar más de un rayo por píxel y luego promediar los resultados.
Esto suaviza las curvas, mitigando el \textit{aliasing}, y provee más muestras aleatorias para la integración de Monte Carlo, lo que reduce el ruido.
Es una técnica muy costosa dado que implica lanzar cantidades mucho mayores de rayos.
Para imágenes complejas, pueden ser necesarios cientos de rayos por píxel para reducir el ruido a niveles aceptables.
Esto imposibilita la generación de imágenes en tiempo real.
Debido a la complejidad computacional agregada por esta técnica, han surgido otras.
La estrella de las técnicas de reducción de ruido son los \textit{denoisers} basados en el uso de redes neuronales.
% TODO: Referenciasss

Varios métodos de iluminación global basados en el trazado de rayos han surgido desde entonces.

\section{Trazado de conos}\label{sec:historical-cone-tracing}

En su artículo de 1984 \cite{ray-tracing-with-cones}, Amanatides propone una extensión al trazado de rayos en la que redefine los rayos por conos.
Los rayos tienen un origen, una dirección y son infinitesimalmente finos.
Los conos tienen a su vez un ángulo de apertura, lo cual agrega un grosor no despreciable.
Usando este grosor, los conos son capaz de no solo probar la existencia de intersecciones, sino también calcular el porcentaje de área de la intersección.
Este método fue propuesto para solucionar el \textit{aliasing} presente en el trazado de rayos, como alternativa al \textit{supersampling}.
En lugar de lanzar muchos rayos por cada píxel, se lanza un solo cono que integra una mayor área de la escena.
Dado que el cono tiene en cuenta la contribución de la luz de varias direcciones dentro de su volumen, reduce el \textit{aliasing} y el ruido al muestrear mayor parte de la escena y reducir la variancia del muestreo.

Esta idea de trazar conos en lugar de rayos no solo ayuda en el \textit{aliasing} y el ruido.
También permite generar sombras suaves.
En el trazado de rayos, al probar si un punto se encuentra en sombra o no, se lanza un rayo hacia la fuente de la luz.
Si este interseca con algún objeto antes de llegar a la luz, el punto está en sombra.
Esta respuesta binaria, si o no, a la pregunta de si el punto se encuentra en sombra resulta en sombras duras.
Al trazar un cono en lugar de un rayo hacia la fuente de luz, es posible responder el porcentaje de oclusión de ese punto, lo cual lleva a una escala de grises y a sombras suaves.
En la figura \ref{fig:shadow-rays-and-cones} se pueden ver diagramas mostrando esta diferencia.

\begin{figure}
    \begin{center}
    \begin{subfigure}{.49\textwidth}
        \includegraphics[width=\textwidth]{shadow-ray-diagram}
        \caption{Rayo de sombra}
    \end{subfigure}
    \begin{subfigure}{.49\textwidth}
        \includegraphics[width=\textwidth]{shadow-cone-diagram}
        \caption{Cono de sombra}
    \end{subfigure}
    \caption{Los conos de sombra no solo devuelven si hay intersección, también devuelven el porcentaje de área de la intersección}
    \label{fig:shadow-rays-and-cones}
    \end{center}
\end{figure}

Esta técnica sigue hallando la intersección de manera analítica, por lo que las ecuaciones son mucho más complejas.
Tanto es así que originalmente Whitted había considerado usar conos pero la complejidad añadida de las ecuaciones lo hizo descartarlos.
Aún así, los beneficios superan a las desventajas en la mayoría de los casos.

Para calcular la luz indirecta se halla la radiancia saliente de un punto en dirección a la cámara.
En \textit{path tracing}, este valor se calcula aproximando la integral sobre el hemisferio con el método de Monte Carlo.
Esto implica tomar varias muestras, que son los rayos que son lanzados desde $p$ en todas las direcciones $\omega_i$.
En el caso de trazado de conos, en lugar de rayos, se particiona el hemisferio en secciones que pueden ser aproximadas mediante estos conos, como se muestra en la figura \ref{fig:partitioned-hemisphere-cones}.

% TODO: Estaría bueno hablar del método de radiosidad antes capaz
% TODO: Se iba a llamar "Otros métodos de iluminación global" pero por ahora es uno solo jeje
\section{Photon Mapping}\label{sec:photon-mapping}

\textit{Photon Mapping}, propuesto por Henrik Wann Jensen en 2001 \cite{photon-mapping}, es un algoritmo basado en el trazado de rayos que es capaz de simular de manera más realista la refracción de la luz a través de sustancias transparentes como vidrio o agua. 
Funciona ``emitiendo'' fotones de la fuente de luz y almacenando en un mapa de fotones la ubicación de cada interacción de estos con las superficies no especulares ni transparentes de la escena.

Con \textit{photon mapping} se pueden generar cáusticas, que son los dibujos que se generan cuando una superficie especular o transparente concentra la luz en una superficie difusa.
En la figura \ref{fig:caustics} se puede ver este efecto.

% TODO: Estaría bueno poder usar imágenes generadas por nosotros en computación gráfica, pero no guardamos las imágenes en el repo como hicimos con whitted :c
\begin{figure}[h!]
    \begin{subfigure}{.5\textwidth}
        \centering
        \includegraphics[width=\textwidth]{photon-mapping-cornell-box-torus.png}
        \caption{Con vidrio}
    \end{subfigure}
    \begin{subfigure}{.5\textwidth}
        \centering
        \includegraphics[width=\textwidth]{photon-mapping-cornell-box-water.png}
        \caption{Con agua}
    \end{subfigure}
    \caption{Cáusticas. Fuente: \cite{faster-photon-mapping}}
    \label{fig:caustics}
\end{figure}

El algoritmo comienza con una etapa en la que se lanzan fotones desde la fuente de luz hacia la escena.
Estos fotones se almacenan en las superficies difusas de la escena, creando un \textbf{mapa de fotones}.
Estos fotones se usan en una segundo etapa cuando se está calculando el color de un píxel.
Además de los rayos de ray tracing de reflexión y refracción, se lanzan rayos adicionales en direcciones aleatorias que buscan en el mapa de fotones, simulando la reflexión difusa.
Es una extensión a ray tracing, que utiliza el paso adicional de lanzado y evaluación de fotones.

Al igual que con el trazado de rayos, varias mejoras y optimizaciones han surgido a lo largo de los años.
Variantes de la técnica original han sido desarrolladas haciendo uso de tarjetas gráficas, alcanzando el tiempo real \cite{real-time-photon-mapping}.
Imágenes generadas por esta técnica pueden verse en la figura \ref{fig:real-time-photon-mapping}.

\begin{figure}
    \begin{subfigure}{.5\textwidth}
        \centering
        \includegraphics[width=\textwidth]{real-time-photon-mapping-direct-only.png}
        \caption{Solo luz directa}
    \end{subfigure}
    \begin{subfigure}{.5\textwidth}
        \centering
        \includegraphics[width=\textwidth]{real-time-photon-mapping-indirect-only.png}
        \caption{Solo luz indirecta}
    \end{subfigure}
    \begin{subfigure}{\textwidth}
        \centering
        \includegraphics[width=\textwidth]{real-time-photon-mapping.png}
        \caption{Todo junto}
    \end{subfigure}
    \caption{Photon mapping en tiempo real. Fuente: \cite{real-time-photon-mapping}}
    \label{fig:real-time-photon-mapping}
\end{figure}

\section{Vóxeles}\label{sec:voxels}

Un vóxel es el equivalente de un píxel en el espacio 3D.
Así como un píxel es un elemento de imagen, \textit{\textbf{pi}cture \textbf{el}ement}, un vóxel es un elemento de volumen, \textit{\textbf{vo}lume \textbf{el}ement} \cite{rtr}.
Similar a como los píxeles se ubican en una grilla que divide una superficie 2D en secciones cuadradas, los vóxeles dividen un volumen 3D en cubos.

Tradicionalmente, se usan para guardar datos volumétricos y como primitiva para renderizar una variedad de objetos.
Permiten representar volúmenes, en contraposición al triángulo, que se utiliza para representar únicamente superficies.
Son un buen candidato para renderizar volúmenes y modelos 3D como el humo, la niebla, el fuego, los huesos y el terreno, entre otros.

Algunos de estos elementos se ven en la figura \ref{fig:voxels_for_rendering}. % TODO: Encontrar fotos más nuevas, estas se ven que son viejasas

\begin{figure}
    \begin{center}
        \begin{subfigure}{.49\textwidth}
            \centering
            \includegraphics[width=\textwidth]{voxel-smoke.png}
            \caption{Vóxeles representando humo. Fuente: \cite{voxel-smoke}}
        \end{subfigure}
        \begin{subfigure}{.49\textwidth}
            \centering
            \includegraphics[width=\textwidth]{voxels-for-trees.png}
            \caption{Vóxeles representando vegetación. Fuente: \cite{voxels-for-trees}}
        \end{subfigure}
    \end{center}
    \caption{Vóxeles para renderizar humo y vegetación}
    \label{fig:voxels_for_rendering}
\end{figure}

% TODO: Completar con papers
% Los vóxeles también son utilizados para visualizaciones científicas.

% Además de primitiva de renderizado, los vóxeles se utilizan cada vez más para manipular datos espaciales, como en NST (\textit{Neural Style Transfer}) \cite{transport-based-neural-style-transfer}, que fue utilizado en la reciente película de Pixar \textit{Elemental} \cite{elemental-neural-style-transfer}.

Cada vóxel marca si la zona del espacio que representa está ocupada o libre, y por ejemplo en contextos médicos se usan para indicar la opacidad y densidad de un hueso.
Para el renderizado, se pueden utilizar para almacenar valores como el color o la irradiancia en cada vóxel.
En general, no es necesario guardar la posición de un vóxel, ya que su lugar en la grilla es lo que indica su posición.

El proceso de convertir otra representación, por ejemplo una malla de polígonos, en una estructura de vóxeles se llama voxelización.
Esto involucra intersecar la representación con la grilla de vóxeles, y marcar como ocupados los vóxeles que se solapen con esta.

Algunos programas usan grillas completas de vóxeles.
Sin embargo, en la mayoría de los casos no es necesario.
Una observación útil es que en muchas aplicaciones es suficiente tener vóxeles en el límite entre un objeto y el espacio vacío, siendo innecesario el relleno.

Otra observación útil es que, para una buena representación, es suficiente con tener mucho detalle en la frontera entre espacio vacío y lleno.
Para lograr esto, se puede usar un \textit{octree} disperso, que será explicado a continuación.

\section{Octrees}\label{sec:octree}

Un \textit{octree} es una estructura de datos de ``árbol'', en la que cada nodo interno (no hoja) tiene $8$ hijos \cite{rtr}.
Suele usarse para representar datos espaciales \cite{octree-textures}.
Se puede utilizar para dividir el espacio 3D de manera jerárquica, con varios niveles.

Se parte de un volumen original cúbico o paralelepípedo que se divide en dos partes iguales por dimensión, resultando en $8$ octantes iguales.
En la estructura de árbol, el nodo raíz representa el volumen original y cada uno de sus $8$ hijos representa a cada octante.
Aplicando recursivamente, se divide el espacio en $8^{(n - 1)}$ secciones, donde $n$ es la cantidad de niveles del árbol, y el primer nivel tiene un solo nodo que representa toda la escena.

De manera similar, un espacio 2D se puede dividir utilizando un \textit{quadtree}, en donde cada nodo interno tiene $4$ hijos.
Los \textit{quadtrees}, al ser bidimensionales, son más fáciles de visualizar, por lo que serán utilizados a lo largo de este informe para explicar aspectos que funcionan igual tanto en ellos como en su equivalente tridimensional.
En la figura \ref{fig:quadtree} se muestra un \textit{quadtree} en su forma de árbol y en el espacio que subdivide.

\begin{figure}
    \begin{subfigure}{\textwidth}
        \centering
        \includegraphics[width=.4\textwidth]{quadtree.png}
        \caption{Visto en su estructura de árbol, con punteros a cada hijo}
    \end{subfigure}
    \begin{subfigure}{\textwidth}
        \centering
        \includegraphics[width=.6\textwidth]{quadtree-spatial.png}
    \caption{Visto espacialmente}
    \end{subfigure}
    \caption{Quadtree}
    \label{fig:quadtree}
\end{figure}

Cuando la naturaleza de la información lo permite, se puede evitar subdividir un nodo del árbol si sus $8$ hijos tienen todos la misma información.
De esta idea surge el \textit{octree} disperso.

A la hora de voxelizar una escena, los vóxeles pueden ubicarse dentro de un \textit{octree} disperso, en lugar de en una grilla.
En este caso, los nodos no se subdividen si no hay geometría dentro de la región del espacio que representan, dado que no hay vóxeles en esa región.

% TODO: Imagen de esto

\section{Ducto gráfico}\label{sec:graphics-pipeline}

Dada una escena 3D, una cámara virtual, varios objetos y varias fuentes de luz, ¿cómo se genera una imagen 2D en el monitor de una computadora?

Si bien es posible generar una imagen a partir de una escena usando la CPU, las GPUs, o tarjetas gráficas, están especialmente diseñadas para esto.

Las tarjetas gráficas ejecutan un \textbf{ducto gráfico} para generar las imágenes.
Esto es, una secuencia de transformaciones y operaciones que parten de primitivas y generan la imagen final.
Existen varios tipos de ductos gráficos, el ducto raster, el de cómputo de propósito general, el de trazado de rayos, entre otros.
El ducto raster es importante, dado que es utilizado en \textit{voxel cone tracing}.

El ducto raster parte de vértices de mallas poligonales, aplica transformaciones, realiza pruebas de profundidad y calcula el color de cada píxel de la imagen final.

Cada paso de un ducto gráfico puede ser fijo o programable.
En el caso de que sea fijo, el mismo ya está implementado en la tarjeta gráfica.
En algunos casos, estos pasos fijos proveen parámetros para configurar su comportamiento, este es su mayor grado de libertad.
En el caso de que sea programable, se puede escribir un \textit{shader}, un programa de sombreado que corre en la GPU, que implementa el paso en su totalidad.
Esto aporta un gran grado de libertad a la hora de renderizar escenas.

Existen varias APIs, interfaces, con las que se puede interactuar con una tarjeta gráfica \cite{comparison-graphics-apis}, notablemente Vulkan, Direct3D, Metal, WebGPU y OpenGL.
Todas estas permiten acceder al ducto de raster.

En OpenGL, por ejemplo, este ducto posee las etapas que se ven en la figura \ref{fig:raster-pipeline} \cite{rtr}.

\begin{figure}[h!]
    \centering
    \includegraphics[width=\textwidth]{raster-pipeline.png}
    \caption{Ducto de raster. Fuente: \cite{rtr}}
    \label{fig:raster-pipeline}
\end{figure}

De estas, tres son programables.
El desarrollador debe escribir \textit{shaders} para estos pasos, que se pueden escribir en el lenguaje GLSL \cite{glsl-spec}.
Los \textit{shaders} son programas que ejecutan en la GPU.
Estos programas son ejecutados con un alto grado de paralelización en la tarjeta gráfica.

Las etapas programables son el \textit{vertex shader} (\textit{shader} de vértices), \textit{geometry shader} (geometría), y \textit{fragment} o \textit{pixel shader} (fragmentos o píxeles).
Cada una de estas etapas pueden comunicar datos a etapas posteriores.
El resto de las etapas son fijas.

El \textit{vertex shader} toma los vértices de la geometría de la escena y los puede transformar a otro sistema de coordenadas.
En general es usado para pasar los vértices de espacio local de coordenadas a espacio global, de vista y luego proyección.
Esto se logra usando tres matrices que se conocen como modelo, vista y proyección.
Un hilo es ejecutado por cada vértice de las primitivas de entrada.

El \textit{geometry shader} puede generar nuevos vértices, por lo que es útil para agregar complejidad extra a la geometría de la escena.
Aquí se ejecuta un hilo por cada primitiva de salida del \textit{vertex shader}, pero estas primitivas pueden tener más de un vértice.

Finalmente, el \textit{fragment shader} (o \textit{pixel shader}) trabaja con píxeles y no con vértices.
Es en este \textit{shader} donde se realizan los cálculos de iluminación para calcular el color de cada píxel.
Se ejecutan mínimo un hilo por cada píxel de la imagen que se quiere generar.
Pueden ejecutarse más de uno en el caso que dos objetos aporten color al mismo píxel, en cuyo caso puede ser que uno sea descartado o pueden mezclarse los colores de ambos en caso de que sean transparentes.

Otro ducto muy relevante, que es usado para implementar la mayoría del algoritmo, es el ducto de cómputo de propósito general.
Este es muy simple, consiste en la inicialización de datos de entrada, luego la ejecución de un programa en la GPU llamado \textit{compute shader} (\textit{shader} de cómputo) y finalmente el retorno de datos hacia la CPU.
Notar que este ducto no es utilizado para generar una imagen, sino para realizar cálculos arbitrarios que toman una entrada y producen una salida, aprovechando la alta paralelización que provee la tarjeta gráfica.

% TODO: Agregar una referencia acá
\section{Renderizado diferido}\label{sec:deferred-rendering}

A la hora de renderizar una escena, hay dos grandes modelos que se pueden utilizar: el clásico, conocido como \textit{forward rendering} y una opción alternativa conocida como renderizado diferido.

En el primero, se procesan todos los vértices en el programa.
Cada uno pasa por cada etapa del ducto gráfico y aporta a la imagen final a menos que sea desechado por un test de profundidad.

En el renderizado diferido, se procesan todos los vértices del programa solo para la etapa del \textit{vertex shader}.
Se guarda toda la información necesaria para etapas posteriores en texturas llamadas \textit{geometry buffers}.
Al hacer esto, se ahorran los cálculos para todos los vértices que son desechados después del vertex shader. % TODO: Es durante o después
% TODO: Hablar del volumen de vista?
Estos suelen ser un buen porcentaje del total, teniendo en cuenta los objetos fuera del ángulo de vista y detrás de otros objetos.

El \textit{forward rendering} es conceptualmente más sencillo y más fácil de implementar.
En escenas con pocos objetos y fuentes de luz, la complejidad extra del renderizado diferido no está justificada, dado que el aumento de rendimiento es despreciable.
A su vez, \textit{forward rendering} soporta mejor la transparencia.
Sin embargo, a medida que la complejidad de la escena aumenta, el renderizado diferido se vuelve cada vez una opción mejor para mejorar la eficiencia.
También, el renderizado diferido facilita ciertas técnicas de post-procesamiento de la imagen, como bloom, HDR (High Dynamic Range), corrección gamma, y normalización, entre otras.

% \section{Integral de rendering}

% Ya fue mencionada la ecuación de rendering \ref{eq:rendering-equation}.
% % TODO: Esto es teoría de la luz en general, ver cómo ponerlo
% Este modelo se basa en la óptica geométrica, que asume que la luz se propaga en una linea recta cuando no tiene interacción con la materia, en contraste con la óptica física, que considera la característica de onda de la luz y sus dos posibles estados de polarización.

% La energía de luz se describe por su radiancia $I(x, \omega)$.
% Describe el campo de radiación en un punto $x$, dada la dirección $\omega$.
% Se expresa en $W\cdot sr^{-1}\cdot m^{-2}$ y se define como:

% \begin{equation}
%     I(x, \omega) = \frac{dQ}{dA\cdot cos\theta\cdot d\Omega\cdot dt}
% \end{equation}

% con $Q$ la energía radiante (en Joules), $A$ el área (en $m^2$), $\theta$ el ángulo entre la dirección de la luz $\omega$ y el vector normal a la superficie, y $\Omega$ el ángulo sólido (en estereorradianes).

% La radiancia a lo largo de un rayo de luz es afectada cuando pasa por un medio participativo.
% Esta interacción entre la luz y la materia se modela con tres tipos de interacciones: emisión, absorción y dispersión.

% Emisión describe la cantidad de energía radiante de luz que se emite directamente por el medio participativo.
% Absorción es la cantidad de energía que es absorbida por el material.
% Finalmente, la dispersión describe la cantidad de energía que es dispersada por el material, cambiando la dirección de la propagación de la luz.
% La dispersión puede tanto incrementar (dispersión entrante) como reducir (dispersión saliente) la energía a lo largo del rayo.

% La ecuación de la transferencia de luz se obtiene combinando estos tres efectos:

% \begin{equation}
%     \omega \cdot \nabla_x I(x, \omega) = -\chi I(x, \omega) + \eta
% \end{equation}

% donde $\nabla_x I$ es la derivada direccional (el gradiente) de la radiancia.
% El producto escalar entre la dirección de la luz $\omega$ y el gradiente de la radiancia $\nabla_x I$ describe el gradiente tomado en la dirección de la luz.
% El término $\chi(x, \omega)$ es el coeficiente de absorción total.
% Se define como la suma de $\kappa(x, \omega)$, el verdadero coeficiente de absorción, y $\sigma(x, \omega)$, el coeficiente de dispersión saliente:

% \begin{equation}
%     \chi = \kappa + \sigma
% \end{equation}

% La razón $\frac{\sigma}{\chi}$ de coeficiente de dispersión saliente sobre coeficiente de dispersión total es el \textit{albedo}.
% Un albedo de $1$ significa que no hay absorción, hay dispersión perfecta.

% El término $\eta(x, \omega)$ de la ecuación es el coeficiente de emisión total.
% Es la suma del coeficiente de emisión real $q(x, \omega)$ y el coeficiente de dispersión entrante $j(x, \omega)$:

% \begin{equation}
%     \eta = q + j
% \end{equation}

% Mientras $\kappa$, $\sigma$ y $q$ son propiedades ópticas del material, el coeficiente de dispersión entrante $j$ tiene que ser calculado integrando la contribución de todas las direcciones entrantes de luz sobre la esfera:

% \begin{equation}
%     j(x, \omega) = \frac{1}{4\pi} \int_\Omega \sigma(x, \omega')\cdot p(x, \omega', \omega'')\cdot I(x, \omega')\cdot d\omega'
% \end{equation}

% Las contribuciones de la luz incidente $I(x, \omega')$ son pesadas por tanto el coeficiente de dispersión $q$ y una \textit{función de fase} $p(x, \omega', \omega)$ que describe la probabilidad de dispersión desde la dirección entrante $\omega'$ hacia la dirección $\omega$.

% Como usualmente consideramos la transferencia de luz a través de un solo rayo, podemos reescribir la ecuación para considerar un parametro $s$ a lo largo del rayo expresado como $x = p + s\omega$, con $p$ un punto de referencia arbitrario en el rayo:

% \begin{equation}\label{eq:light_transfer}
%     \frac{dI(s)}{ds} = -\chi(s)I(s) + \eta(s)
% \end{equation}

% En vóxel cone tracing, se usa el modelo óptico de emisión-absorción, descrito en \cite{real-time-volume-graphics}.
% En este modelo, se ignora la dispersión y la iluminación indirecta, representando solo emisión y absorción.
% La ecuación \ref{eq:light_transfer} queda:

% \begin{equation}
%     \frac{dI(s)}{ds} = -\kappa(s)I(s) + q(s)
% \end{equation}

% Esta ecuación se puede resolver transformandola en una integral pura entre un punto de comienzo en el rayo $s = s_0$ y un punto de fin $s = D$.
% Definimos $I_0$ como la radiancia inicial.
% La integral resultante es:

% \begin{equation}
%     I(D) = I_0\cdot e^{-\tau(s_0, D)} + \int_{s_0}^D q(s)\cdot e^{-\tau(s, D)}\cdot ds
% \end{equation}

% El término $\tau(s1, s2)$ se llama la profundidad óptica o grosor óptico.

% % TODO: Estaría bueno hablar de todas las referencias que habla el paper acá sobre luz. Completar.
% % Hay hasta un par de papers que usan vóxels, solo un poco distinto a ellos.
% % Cuando arrancó la tésis, tendríamos que habernos mandado de una a leer todas estas referencias.
% Las mejores soluciones en terminos de velocidad trabajan en el espacio de imágen en lugar del espacio del mundo.
% vóxel cone tracing es menos eficiente que esas soluciones pero no requiere aproximaciones tan fuertes.

% % TODO: Volver a poner la parte de normlaes si realmente usamos normales.
% % Ojalá no.

% % \section{Filtrado de normales}\label{sec:normal_filtering}

% % Filtrar datos geométricos, como las normales, no es tan fácil como con colores.
% % El problema, como se muestra en la figura \ref{fig:ndf}, es que dos normales en direcciones opuestas, al ser promediadas, no representan correctamente la región conjunta.
% % Todos los enfoques para filtrar normales se basan en la idea de representar la información de las normales como una distribución estadística de direcciones.
% % Esta representación se define por una función de distribución de normales (NDF, por sus siglas en inglés), que dada una dirección, devuelve la densidad de normales para cada punto de la superficie.

% % \begin{figure}[h!]
% %     \centering
% %     \includegraphics[width=\textwidth]{figs/ndf.png}
% %     \caption{El problema filtrando normales. Fuente: \cite{normal-map-filtering}}
% %     \label{fig:ndf}
% % \end{figure}

% % \cite{toksvig} propuso una representación muy compacta para un NDF.
% % Esta representación se basa en solo una dirección media no normalizada generalmente computada con un mipmap simple de un normal map.
% % El largo de este vector media se usa como una medida de la consistencia de las direcciones normales de la superficie, y se usa para estimar la desviación estándar de una distribución Gaussiana.


% diseño
% La parte central del trabajo refiere a lo que es producción propia o aporte del
% proyecto de grado, incluyendo las decisiones tomadas. Por ejemplo, puede incluir
% los requerimientos, el análisis y el diseño de la solución. Si el proyecto tiene una
% implementación, debe describirse en términos de decisiones tomadas en ese sentido.
% Los detalles de programación se dejan para los anexos.

% Se pueden incluir figuras en el documento, las mismas deben estar referenciadas en el texto. Por ejemplo, la Figura~\ref{fig:logos} muestra los logos de Facultad de Ingeniería y de la Universidad de la República.

% \begin{figure}[h!]
%     \centering
%     \includegraphics[width=\textwidth]{figs/logo-udelar-fing.png}
%     \caption{Logos de FIng y UdelaR}
%     \label{fig:logos}
% \end{figure}

\graphicspath{{chapters/3_diseño/figures/}}

% parte central
\chapter{Voxel cone tracing}\label{chap:design}

En este capítulo se detalla el diseño de \textit{voxel cone tracing} \cite{voxel-cone-tracing}, un algoritmo de iluminación global en tiempo real, que utiliza conceptos presentes en el trazado de rayos (\ref{sec:ray-tracing}), trazado de conos (\ref{sec:historical-cone-tracing}) y \textit{photon mapping} (\ref{sec:photon-mapping}), 
pero reduciendo los costos asociados a estos algoritmos clásicos usando una representación de la escena con vóxeles (\ref{sec:vóxeles}), almacenada en un \textit{octree} disperso (\ref{sec:octree}), para aproximar los conos.

Otra forma de reducir costos y aumentar la velocidad del algoritmo es utilizando el ducto de rasterización \ref{sec:graphics-pipeline} y renderizado diferido \ref{sec:deferred-rendering}.
Los cálculos de luz se realizan en el \textit{fragment shader} del ducto de raster, sobre los \textit{geometry buffers} para reducir la cantidad de fragmentos o hilos de la gpu sobre los que se corren calculos pesados, sin perder calidad de imagen.
Lo anteriormente explicado difiere del trazado de rayos clásico, que renderiza toda la escena puramente mediante intersecciones entre rayos y geometría.

% TODO: Este párrafo es nuevo, así que hay que darle una pasada
La principal aproximación que realiza este algoritmo es trabajar sobre una representación voxelizada pre-filtrada de la escena.
Dadas las coordenadas de un punto de la escena, se recorre el árbol para hallar el vóxel que representa la región del espacio que incluye ese punto.
Este vóxel contiene toda la información necesaria sobre oclusión, color e irradiancia debido a que la estructura es previamente filtrada.
En el filtrado, la información en las hojas del árbol es promediada hacia niveles mayores hasta que cada nivel del árbol contiene información que aproxima las capas inferiores.
La información de un vóxel en cada nivel resume la información de los vóxeles de los nodos hijos que comparten el mismo espacio en la escena.
Se vió en la sección sobre trazado de conos (\ref{sec:ray-tracing}), que en su planteo original, se hallaban las intersecciones del cono con los objetos de la escena de manera analítica, lo que es costoso.
Aquí, en lugar de hallar la intersección analiticamente, se aproxima el cono utilizando la estructura jerárquica de vóxeles lo cual permite acumular la información a lo largo del cono mas eficientemente, pero con cierta perdida de detalle.

% TODO: Estaría bueno poner diagramas de cada etapa:
% - Voxelización: no se qué hacer
% - Construcción y filtrado
% - Inyección de fotones: puedo poner un dibujo de rayos saliendo de la fuente de luz
% - Trazado de conos: poner un cono agarrando varios vóxeles

El algoritmo se puede dividir en 4 grandes etapas:

\begin{enumerate}
    \item Voxelización de la escena
    \item Construcción y filtrado del \textit{octree} disperso
    \item Inyección de fotones
    \item Trazado de conos
\end{enumerate}

La voxelización, la inyección de fotones y el trazado de conos usan el ducto de rasterización, mientras que la construcción del árbol y el filtrado usan el ducto de cómputo de propósito general.
En el resto del capítulo se explicará con mayor grado de detalle cada una de estas etapas.

\section{Voxelización}\label{sec:voxelization}

La escena se divide en una grilla de vóxeles.
La cantidad de vóxeles es configurable, siendo usualmente $512$ o $1024$ por dimensión, lo cual lleva a un total de $512^3$ o $1024^3$ vóxeles respectivamente.

Para voxelizar la escena, se realiza el procedimiento detallado en ``OpenGL Insights, capítulo 22'' \cite{opengl-insights}, un aporte escrito por Crassin luego de haber publicado el artículo de \textit{voxel cone tracing}, aprovechando mejoras en las herramientas disponibles en el momento.
Ese mismo proceso será explicado a continuación, para más información, consultar la referencia.

Usando el ducto de rasterización de OpenGL, se procesan todos los triángulos de la geometría de la escena utilizando como resolución para el rasterizado la resolución de la grilla de vóxeles.
Esto genera una lista de \textbf{vóxeles}, donde cada voxel es una posición dentro de la grilla y un color, el cual es el color de un triangulo que interseca con el espacio representado por el voxel.
Cada uno de estos vóxeles será usado para construir el árbol y terminará almacenado en la estructura. Dado lo anteriormente descrito, el mismo voxel, identificado por su posicion en la grilla, puede estar presente en al lista multiples veces con colores diferentes si mas de un triangulo interseca con el voxel. 

La voxelización de un triángulo $B$ a un vóxel $V$ puede hacerse si:

\begin{enumerate}
    \item El plano de $B$ interseca $V$.
    \item La proyección 2D del triángulo $B$ por la dimensión dominante de su normal (la que provee la mayor área proyectada) interseca la proyección 2D de $V$.
\end{enumerate}

Basado en esta observación, se sigue la serie de pasos que se muestra en la figura \ref{fig:voxelization_pipeline}.

\begin{figure}[h!]
    \centering
    \includegraphics[width=\textwidth]{voxelization_pipeline.png}
    \caption{Ducto de voxelización. Fuente: \cite{opengl-insights}}
    \label{fig:voxelization_pipeline}
\end{figure}

Primero, cada triángulo de la geometría se proyecta ortográficamente en la dimensión dominante de su normal.
La dimensión dominante se elige dinámicamente por triángulo en el \textit{geometry shader}, donde la información de los tres vértices de cada triángulo está disponible,
maximizando el area del triangulo proyectado.

Cada triángulo proyectado se rasteriza para conseguir fragmentos correspondientes a la resolución 2D de la grilla de vóxeles.
Se fija el tamaño del \textit{viewport} a coincidir con la cantidad de vóxeles, por ejemplo un \textit{viewport} de tamaño $512\times512$ para una grilla de $512^3$ vóxeles.
% Mientras se hace esto, se mantienen las operaciones del framebuffer desactivadas, como el depth testing. % TODO: Habría que explicar por qué, no entendí del capítulo de OpenGL Insights. No se por que se desactiva el depth testing, si igual lo que se genera en el frame buffer no se usa, suena como algo de bajo nivel

Durante la rasterización, cada triángulo genera un conjunto de fragmentos 2D. Estos fragmentos corresponderian n a 1 con los vóxeles de la grilla si esta fuese 2D, ya que varios triangulos pueden intersecar con el mismo voxel.
Sin embargo al ser una grilla 3D es necesario calcular la intersección del plano del triangulo con los voxles para encontrar los valores de profundidad en la grilla.
Debido a la elección de la dimensión dominante de la normal para la proyección, solo pueden intersecar con 3 vóxeles en profundidad, por lo que cada fragmento genera de 1 a 3 vóxeles con mismos valores de $x$ e $y$ pero diferentes valores de $z$. % TODO: Por qué? No se explicarlo sin algun dibujo :sweat-smile: Yo tampoco, asi que voy a agregar dibujitos o dejarlo para que nos pregunten porque creo que es de lo que mas me cuesta explicar
Entonces, por cada fragmento 2D, los vóxeles que intersecaron con el triángulo se calculan en el \textit{fragment shader}, basándose en la posición, la profundidad y las derivadas en espacio de pantalla. % TODO: Hacemos lo de las derivadas? Si!

Luego de realizada la voxelización, se obtiene la lista de vóxeles necesaria para crear el árbol, con su posicion y color.
Cada uno tiene una coordenada que lo identifica dentro de la grilla 3D de la escena, así como color y posición en el espacio de la escena.

\subsection{Rasterización conservativa}

El método descrito anteriormente a veces no crea vóxeles para elementos muy finos, como un asta de bandera ya que en el rasterizado solo se prueba el centro del píxel contra los triángulos para generar fragmentos. % TODO: Revisar en el código
Se necesita una manera de generar fragmentos para cada píxel tocado por un triángulo, no necesariamente en el centro.
Un algoritmo así se detalla en \cite{conservative-rasterization}.

La idea es generar, por cada triángulo, un polígono acotante ligeramente más grande, para asegurarse que cualquier triángulo proyectado que interseca con cualquier punto del píxel también interseque con su centro.
Se logra trasladando las aristas del triángulo hacia afuera, la mitad de la diagonal de un píxel, y extendiendolas para generar un triangulo semejante.
A su vez se generan fragmentos nuevos que resultan de sobreestimar la cobertura de este triángulo, dado que este nuevo triangulo puede intersecar con el centro de píxeles que antes no intersecaban en ningun punto.
Para evitar estos fragmentos tambien se genera una caja acotante alineada con los ejes, y se descartan fragmentos por fuera de la misma.
Este proceso se muestra en la figura \ref{fig:conservative_rasterization}.

\begin{figure}[h!]
    \centering
    \includegraphics[width=\textwidth]{conservative_rasterization.png}
    \caption{Rasterización conservativa. Fuente: \cite{opengl-insights}}
    \label{fig:conservative_rasterization}
\end{figure}

\section{\textit{Octree} disperso}

Para almacenar los vóxeles generados, se usa un \textit{octree} disperso, como los vistos en la sección \ref{sec:octree}.
Un octree denso subdivide la escena en 8 cubos y cada cubo se subdivide en 8, y asi sucesivamente. Esta estructura puede escalar rápidamente, generando problemas importante de volumen y velocidad de acceso necesarios de memoria.
Para alivianar este problema se crearon los octree dispersos, donde los nodos no se subdividen si no poseen geometría dentro.

Cada elemento del árbol es un \textbf{nodo}.
Un nodo del árbol representa una sección de la escena.
Cada nivel tiene una cierta cantidad de nodos.
Si el árbol fuera denso, cada nivel $n$ tendría $8^n$ nodos.
El nivel $0$ tendría $1$ nodo, el nivel $1$ tendría $8$, el $2$ $64$ y así sucesivamente. % Esto se desprende de lo anterior, pero me parece una simplificacion posiblemente util
El último nivel del árbol es el que llega a la resolución deseada de $512$ o $1024$ vóxeles.
Valores más altos de resolución crean más niveles del \textit{octree} y hacen que se asemeje cada vez más la aproximación de vóxeles a la geometría real.

% TODO: Podría agregar fotos de la visualización del octree a distintas resoluciones

Dado que la estructura tiene como máximo $512$ o $1024$ vóxeles de resolución, sin importar la geometría de la escena, los cálculos sobre ella son independientes de la complejidad de la geometría.

\subsection{Nodos y bricks}\label{sec:nodes_and_bricks}

Los nodos del árbol no almacenan los vóxeles mencionados en \ref{sec:voxelization}.
Cada nodo almacena únicamente un puntero a sus, como máximo $8$, hijos.

Cada nodo tiene asociado un \textbf{brick}, otra estructura que también representa una región del espacio.
Cada brick está dividido en 27 vóxeles, distribuidos en una estructura de $3\times3\times3$ vóxeles.
Son estos vóxeles los que almacenan los valores de la escena.
En la figura \ref{fig:node_and_brick} se puede observar un nodo con su brick asociado de la manera en la que se disponen en el espacio.
Los bricks ocupan más espacio que sus nodos, ya que los vóxeles se centran en los vértices del nodo.
Lo cual es necesario para que los bricks puedan obtener valores de sus vecinos, lo que garantiza que la interpolación dentro de un solo brick toma en cuenta los valores de sus vecinos resultando en una frontera compartida entre vecinos, como se puede ver en la figura \ref{fig:brick_border_overlap}.
La frontera entre dos nodos adyacentes del mismo nivel son los 9 vóxeles de la cara compartida entre los nodos, los cuales estan presentes en los bricks de ambos nodos, lo cual es necesario para que la interpolación a la hora de generar la imagen final funcione.
Un \textit{shader} llamado \textit{border\_transfer}, es el que se encarga de lograr la coherencia en la frontera entre dos \textit{bricks}.
Se explicará en la Sección \ref{sec:border_transfer}.

\begin{figure}[h!]
    \centering
    \includegraphics[width=.3\textwidth]{node-and-brick.png}
    \caption{Nodo (en rojo) con su brick asociado (azul)}
    \label{fig:node_and_brick}
\end{figure}

\begin{figure}[h!]
    \centering
    \includegraphics[width=.5\textwidth]{brick-border-overlap.png}
    \caption{Solapamiento entre vóxeles de bricks de nodos vecinos}
    \label{fig:brick_border_overlap}
\end{figure}

\subsection{Construcción}\label{design:svo_construction}

Para generar el octree disperso se usa la lista de vóxeles generada durante la voxelización.
Se empieza con un árbol con un solo nivel, con un solo nodo que ocupa toda la escena.
Se ejecuta el algoritmo a continuación sobre cada nivel para generar el siguiente hasta completarlo.

% TODO: Dibujos para explicar flag y allocate

Dado un nivel $i$ del árbol, dos programas principales son ejecutados en secuencia para generar el nivel $i + 1$: \textit{flag\_nodes} y \textit{allocate\_nodes}.

Se corre un hilo de \textit{flag\_nodes} por cada vóxel de la lista de vóxeles.
Dado un vóxel, se recorre el árbol construido hasta el momento, hasta que se llega a una sección del nodo no subdividida y se marca para ser subdividida.

Luego, se ejecuta \textit{allocate\_nodes}, que busca en el nivel $i$ las secciones marcadas para subdividir.
Al encontrar una sección de un nodo marcada, crea un nuevo nodo en la estructura y cambia la marca por un puntero a ese nuevo nodo.

Siempre y cuando haya geometria en la región de la escena representada por un nodo, la misma será subdividido nivel tras nivel hasta el maximo determinado por la resolución de la grilla de vóxeles.

Una vez alcanzado el último nivel, se escriben los atributos de los vóxeles en los bricks de las hojas del árbol, promediando cuando más de uno tiene la misma posición en la grilla de vóxeles.
Mientras más triángulos tenga la escena original y menos resolución tiene la grilla de vóxeles, mas ocurre lo anteriormente mencionado.
Las hojas no tienen hijos.

% Esto creo que deberia ir en implementacion
En \ref{sec:nodes_and_bricks}, se vió como los bricks ocupan una región más amplia del espacio que su nodo correspondiente.
% TODO: Habría que explicar que los vóxeles son un octavo de nodo del último nivel. Para que esto tenga sentido.
Para consolidar esto con el tamaño de los vóxeles, estos se almacenan únicamente en las esquinas de los bricks. % TODO: Esto esta muy mal explicado
Se almacenan en la esquina más cercana a la posición del vóxel.
Luego, se aplica un programa \textit{spread\_leaves}, para esparcir los valores de las esquinas a lo largo de todo el brick.
Su funcionamiento se muestra en 2D en la figura \ref{fig:spread-leaves}:
\begin{itemize}
    \item El vóxel central almacena el promedio de todas las 8 esquinas
    \item El vóxel del medio de cada cara almacena el promedio de las 4 esquinas de su cara
    \item El vóxel del medio de cada arista almacena el promedio de las 2 esquinas de esa arista
\end{itemize}
Como resultado, se esparcen los valores de las esquinas a todo el brick.
Lo anteriormente descrito es el algoritmo que expande los vóxeles generados poder llenar los bricks $3\times3\times3$.
A partir de aquí, el termino ``vóxel'' refiere a una de las $27$ subdivisiones de un brick, dado que ya se procesó la lista generada durante la voxelización.

\begin{figure}
    \begin{subfigure}{.5\textwidth}
        \centering
        \includegraphics[width=.75\textwidth]{spread-leaves-edge.png}
        \caption{En una arista}
    \end{subfigure}
    \begin{subfigure}{.5\textwidth}
        \centering
        \includegraphics[width=.75\textwidth]{spread-leaves-center.png}
        \caption{En el centro}
    \end{subfigure}
    \caption{Funcionamiento de \textit{spread\_leaves} en 2D. Las flechas indican aporte al promedio}
    \label{fig:spread-leaves}
\end{figure}

\subsection{Border transfer}\label{sec:border_transfer}

Como se mostro en la figura \ref{fig:brick_border_overlap}, las fronteras entre bricks son compartidas, correspondiendo al mismo espacio en la escena.
Por lo tanto, los valores almacenados en esos vóxeles deben ser los mismos garantizando una correcta interpolación a la hora de generar la imagen final.
Sin embargo, luego de aplicar \textit{spread\_leaves}, cada frontera tiene valores distintas en cada par bricks vecinos.
Para igualarlos, se promedian los valores de la frontera con la del brick vecino, asegurandose que el nivel sea coherente.
El shader \textit{border\_transfer} se encarga de promediar los valores en la frontera de cada brick con la de sus vecinos, en X, Y y Z.
De esta manera, aún cuando un vóxel puede estar en varios bricks, en $8$ como máximo, su valor va a ser siempre el mismo en cada uno de ellos. % Creo que tiene razon Eduardo, pero no hay ganas de cambiarlo

% TODO: Dibujos para explicar border transfer

\subsection{Nodos frontera}

Al usar un octree disperso, no existen nodos donde no hay geometría.
\textit{Border transfer} asume que todo nodo tiene un vecino en su mismo nivel en el árbol, lo cual no siempre es verdad.
Por ejemplo en el límite entre la geometría y el espacio vacío existen nodos sin vecino, ya que al ser un árbol disperso no hay nodos donde hay exclusivamente espacio vacio.
Los nodos sin vecino causan un problema, dado que no se puede promediar el valor de sus vóxeles con el vecino, lo que causa problemas de interpolación.
Para que la interpolación continue y logre difuminar el sombreado, es necesaria una capa de nodos extra, los que llamaremos \textbf{nodos frontera}, que representa el espacio vacío adyacente a la geometría.

Los nodos frontera se añaden en cada nivel del árbol a la hora de construírlo.
Sus bricks no contienen valores, existen solo para interpolar los valores con 0 y así difuminar los bordes de la geometría para mejorar la interpolación.
De esta forma los vóxeles compartidos entre espacio vacio y geometría tienen un promedio de ambas, similar al resto de los vóxeles compartidos entre nodos.
El mismo problema no se genera entre estos nodos frontera y el espacio vacio, dado que los vóxeles compartidos con el espacio vacio van a tener como valor la opacidad nula, por lo que se tiene una estructura consistente.

% TODO: Dibujos para explicar nodos frontera

\subsection{Filtrado}\label{design:filtering}

Una vez que todos los atributos se encuentran en las hojas del octree, los mismos seran filtrados a posiciones superiores.
Filtrarlos implica promediarlos de tal manera que para un nodo interior (no hoja) $A$, su brick tenga un promedio de la información contenida en los bricks de todos sus hijos.
El proceso se realiza en $n - 1$ pasos, siendo $n$ el nivel máximo del octree.
En cada paso, se calcula el valor de cada vóxel del brick del padre, usando los bricks de los hijos.

Consideremos un nodo en el penúltimo nivel, con su brick asociado y sus hijos, como muestra la figura \ref{fig:node_with_children}.
En la figuran se muestran solo $4$ hijos porque se usa como ejemplo un cuadtree, la versión 2D del octree, ya que es mas facil de visualizar.
Cada hijo tiene a su vez su propio brick asociado como se muestra en la figura \ref{fig:all_child_bricks}.

\begin{figure}[h!]
    \centering
    \includegraphics[width=.3\textwidth]{node-with-children.png}
    \caption{Nodo con su brick asociado y sus hijos}
    \label{fig:node_with_children}
\end{figure}

\begin{figure}[h!]
    \begin{center}
        \begin{subfigure}{.24\textwidth}
            \includegraphics[width=\textwidth]{first-child-brick.png}
        \end{subfigure}
        \begin{subfigure}{.24\textwidth}
            \includegraphics[width=\textwidth]{second-child-brick.png}
        \end{subfigure}
        \begin{subfigure}{.24\textwidth}
            \includegraphics[width=\textwidth]{third-child-brick.png}
        \end{subfigure}
        \begin{subfigure}{.24\textwidth}
            \includegraphics[width=\textwidth]{fourth-child-brick.png}
        \end{subfigure}
    \end{center}
    \caption{Bricks asociados a cada hijo del nodo}
    \label{fig:all_child_bricks}
\end{figure}

Los valores de los vóxeles del brick padre se calculan en 4 etapas distintas, dependiendo de dónde se ubican en el \textit{brick}: Esquinas, bordes, caras, centro.
Cada una de las etapas calcula un valor parcial para un tipo de vóxel.
Es parcial porque para los vóxeles limítrofes con otro nodo, este valor tiene que luego ser agregado con el de los vecinos, con un tipo de \textit{border\_transfer}.

% El cálculo para cada etapa es el mismo para todos los vóxeles dentro de su grupo, por lo que se mostrará únicamente el calculo para un vóxel de cada grupo: esquinas, bordes, caras, y centro.
% En el caso 2D, no existe el vóxel centro.

\begin{figure}
    \centering
    \includegraphics[width=.5\textwidth]{brick-voxel-naming.png}
    \caption{Una posible forma de referirse a cada vóxel de un brick}
    \label{fig:brick-voxel-naming}
\end{figure}

Dado el vóxel superior izquierdo de la figura \ref{fig:svo_filtering_corners}, se considera solo el brick del hijo superior izquierdo del nodo.
De ese brick, se consideran los vóxeles amarillos en la figura.
El valor final del vóxel del padre se calcula promediando los valores de los vóxeles del hijo, pesados por el porcentaje de solapamiento.
Si nombramos los vóxeles del brick hijo $a, \cdots, i$ y los del brick padre $a', \cdots, i'$, como en la figura \ref{fig:brick-voxel-naming}, entonces el valor del vóxel del padre se calcula como:

$$
a' = a + b * \frac{1}{2} + d * \frac{1}{2} + e * \frac{1}{4}
$$

En el caso tridimensional, hay que agregar un quinto factor multiplicado por $\frac{1}{8}$.

% Esto resulta en un kernel gaussiano. % TODO: Alguna referencia sobre kernels gaussianos? Si no no lo mencionamos

\begin{figure}
    \centering
    \includegraphics[width=.25\textwidth]{svo-filtering-corner.png}
    \caption{
        Filtrado para un vóxel esquina.
        Se puede ver el vóxel del brick padre cuyo valor se quiere calcular, junto con el brick del hijo y su solapamiento con este vóxel.
    }
    \label{fig:svo_filtering_corners}
\end{figure}

De la misma manera se calculan los vóxeles de los bordes y de las caras, solo que en esos casos se usan más de un brick hijo, como se puede ver en las figuras \ref{fig:svo_filtering_edges} y \ref{fig:svo_filtering_faces}.

\begin{figure}
    \begin{center}
        \begin{subfigure}{.24\textwidth}
            \includegraphics[width=\textwidth]{svo-filtering-edge-1.png}
        \end{subfigure}
        \begin{subfigure}{.24\textwidth}
            \includegraphics[width=\textwidth]{svo-filtering-edge-2.png}
        \end{subfigure}
    \end{center}
    \caption{Filtrado para un vóxel borde}
    \label{fig:svo_filtering_edges}
\end{figure}

\begin{figure}
    \begin{center}
        \begin{subfigure}{.24\textwidth}
            \includegraphics[width=\textwidth]{svo-filtering-face-1.png}
        \end{subfigure}
        \begin{subfigure}{.24\textwidth}
            \includegraphics[width=\textwidth]{svo-filtering-face-2.png}
        \end{subfigure}
        \begin{subfigure}{.24\textwidth}
            \includegraphics[width=\textwidth]{svo-filtering-face-3.png}
        \end{subfigure}
        \begin{subfigure}{.24\textwidth}
            \includegraphics[width=\textwidth]{svo-filtering-face-4.png}
        \end{subfigure}
    \end{center}
    \caption{Filtrado para un vóxel cara}
    \label{fig:svo_filtering_faces}
\end{figure}

% TODO: Solo traer de vuelta si realmente usamos normales en el código.
% Si los vóxeles almacenan normales, estas se promedian como se mencionó en la sección \ref{sec:normal_filtering}.

\subsection{Filtrado anisotrópico}

El filtrado descrito en la sección anterior resulta en vóxeles con valores independientes del punto de vista.
Una característica deseable es que los valores de oclusión y opacidad dependan del punto de vista del observador, lo cual se logra en este caso guardando mas de un valor por vóxel para estas propiedades.
Es muy útil a la hora de representar un atributo en una escena 3D.

Para ilustrar el punto anterior, consideremos una escena compuesta únicamente por una pared fina.
Dado un vóxel de un nodo en un nivel alto del árbol, el nodo representa una gran región del espacio.
La región contiene únicamente una pared que pasa por su centro, y el resto de la escena es espacio vacío.
Con el filtrado presentado en las secciones anteriores, llamado \textbf{isotrópico}, el vóxel tendrá un unico valor de opacidad que, al representar un espacio mayormente vacío, tendra un valor bajo.
Sin embargo, se observa que la percepción de la pared es muy distinta si se la ve de frente o de costado.
Como se muestra en la figura \ref{fig:anisotropic-thin-wall}, la pared vista de frente es opaca, pero vista de costado su superficie visible es practicamente nula, lo cual se asemeja a una superficie transparente.

La solución propuesta por Crassin \cite{voxel-cone-tracing} consiste en realizar el filtrado 6 veces por vóxel, uno por cada dirección alineada con los ejes: X, -X, Y, -Y, Z, -Z; y tener en cuenta la dirección a la hora de promediar los valores.
Cada uno de los 6 valores representa el voxel visto desde una de las 6 direcciones previamente mencionadas. Por lo tanto si se observa el vóxel desde una direccion arbitraria, se calcula su oclusión y color a través de la interpolación de las tres direcciones mas cercanas.
Este tipo de filtrado se denomina \textbf{anisotrópico}, dado que depende de la dirección con que se observe al vóxel.

\begin{figure}
    \centering
    \includegraphics[width=.5\textwidth]{anisotropic-thin-wall.png}
    \caption{
        Vóxel que contiene una pared fina.
        En la izquierda, la pared se ve de frente.
        En la derecha, se ve de costado.
        Se espera que el valor del vóxel refleje esta diferencia entre las direcciones de vista.
    }
    \label{fig:anisotropic-thin-wall}
\end{figure}

Dada una dirección, por ejemplo, de izquierda a derecha, se parte de los vóxeles de la izquierda y se calcula un valor para cada fila, partiendo de estos y yendo hacia los vóxeles de la derecha.
Llamémosle al valor de cada fila \textbf{valor direccional}.
En la figura \ref{fig:svo_filtering_anisotropic} se pueden ver todos los valores direccionales que deben ser calculados para un brick en 2D, para todas las direcciones.
Para cada fila, se ejecuta un algoritmo de acumulación de opacidad que va avanzando en la dirección dada.
Si el algoritmo llega a opacidad 1, termina y devuelve el valor direccional para esa fila.

\begin{figure}
    \begin{center}
        \begin{subfigure}{.24\textwidth}
            \includegraphics[width=\textwidth]{anisotropic-filtering-x.png}
        \end{subfigure}
        \begin{subfigure}{.24\textwidth}
            \includegraphics[width=\textwidth]{anisotropic-filtering-x-neg.png}
        \end{subfigure}
        \begin{subfigure}{.24\textwidth}
            \includegraphics[width=\textwidth]{anisotropic-filtering-y.png}
        \end{subfigure}
        \begin{subfigure}{.24\textwidth}
            \includegraphics[width=\textwidth]{anisotropic-filtering-y-neg.png}
        \end{subfigure}
    \end{center}
    \caption{Filtrado anisotrópico en todas las direcciones}
    \label{fig:svo_filtering_anisotropic}
\end{figure}

El filtrado anisotrópico soluciona situaciones como la de la pared mencionada anteriormente.
Para el caso de la pared fina, el vóxel que contiene la pared es opaco en la dirección paralela a la normal de la pared y prácticamente transparente en cualquiera de las perpendiculares.

\section{Inyección de fotones}\label{sec:photon-injection}

Hasta ahora tenemos la estructura de datos creada, conteniendo el color y opacidad de toda la escena en las hojas y un promedio de los niveles inferiores en todos los nodos interiores.
El objetivo del algoritmo es la iluminación global de una escena, por lo que necesitamos información de la luz.
En este paso, lanzamos fotones a partir de la fuente de luz de la escena, similar a como se hace en \textit{photon mapping}, lo cual se logra utilizando el ducto de rasterización.

Se rasteriza la escena desde el punto de vista de la luz para generar una textura 2D.
En lugar de tener colores en cada téxel de la textura, se almacenan las posiciones de los objetos de la escena.
La existencia de una posición en un texel de la textura significa que esa posición es visible desde la luz, por lo cual debe recibir un fotón.
Las posiciones son utilizadas para recorrer el octree y almacenar los fotones en vóxeles de sus hojas.

Los fotones que se almacenan en los vóxeles del árbol, la \textbf{irradiancia}, es el flujo lumínico recibido por cada superficie.
Luego pasan por el mismo proceso de \textit{border\_transfer} y filtrado que el color.
El \textit{border\_transfer} suma en lugar de promediar en este caso, dado que ambos lados de la frontera aportan a la cantidad de fotones total.
El filtrado funciona de la misma manera.

La etapa de construcción debe realizarse solo una vez, mientras que, al soportar luces dinámicas, esta etapa debe ejecutarse cada vez que la luz se mueva.
En esos casos, toda la irradiancia del árbol vuelve a cero y se vuelve a correr el programa que lanza los fotones, y lo llamaremos la actualización de la estructura.

% TODO: Quedó extremadamente corta esta parte
% Estaría bueno hablar de las optimizaciones? Te hacen perder un poco

\section{\textit{Cone tracing}}\label{sec:cone_tracing}

Como se comentó brevemente al inicio del capítulo, el trazado de conos se realiza en el \textit{fragment shader} del ducto de rasterización.
La entrada al algoritmo de \textit{cone tracing} son los \textit{geometry buffers} que contienen los valores de la escena, ya habiendo descartado los vértices fuera de vista.
El algoritmo de trazado de conos es ejecutado para cada píxel de los geometry buffers para calcular el color final.

Dado un punto de origen, se lanza uno o varios conos con cierta dirección y apertura, dependiendo del efecto que se quiere lograr.
% TODO: Tema de las normales todo considerando el hemisferio hacia el lado de la normal del punto.

Para cada cono, se parte desde su origen y se avanza en la dirección de su eje tomando pasos de cierto tamaño.
Esto se conoce como \textit{ray marching}. % TODO: No se si está bueno poner el nombre sin haberlo mencionado antes
Después de cada paso tomado, se calcula el diámetro del cono en ese punto $P$.
Luego se encuentra el nivel del octree con el menor tamaño posible de voxel que cumpla la condición de ser mayor que el diámetro del cono.
Dado ese nivel y la posición del punto $P$, se recorre el octree y se encuentra el nodo que correspondiente a ese nivel que incluye el punto.
Ese nodo tiene un brick asociado, cuyos vóxeles tienen los valores prefiltrados, conseguidos en \ref{design:filtering}.
Se usa el valor del vóxel que corresponde con la posición y se acumula. En caso de vóxeles anisotrópicos, también depende de la direccion del eje del cono.
Se sigue avanzando paso a paso acumulando valores hasta satisfacer un criterio de parada.
El algoritmo de \textit{cone tracing} en si es simple dados todos los pasos anteriores.

Cada cono calcula un color $c$ y una opacidad $\alpha$.
Si en cada paso consideramos $c$ y $\alpha$ como los valores hasta el momento, y $c_2$ y $\alpha_2$ como los valores nuevos encontrados en el vóxel del paso, entonces en cada paso los valores de $c$ y $\alpha$ se calculan de la siguiente manera:

$$
\begin{cases}
    c = \alpha c + (1 - \alpha) \alpha_2 c_2 \\
    \alpha = \alpha + (1 - \alpha) \alpha_2
\end{cases}
$$

Para la luz indirecta difusa, se lanzan conos para cubrir el hemisferio centrado en la normal del punto.
En la mayoría de los casos, 5 conos anchos difusos dan un buen resultado.
Cada cono acumula el color de los vóxeles con los que se encuentra multiplicado por la cantidad de fotones.
Esto logra un efecto de \textit{light bleed}, donde las superficies adquieren color de otras superficies cercanas que reciben y dispersan luz.

Para la luz indirecta especular, se lanza un solo cono fino en la dirección de reflexión.
El cono, al ser más fino, es rápidamente ocluido por nodos de niveles más bajos, con lo que el reflejo tiene mejor definición.
Si se utiliza un mayor ángulo de apertura del cono, el reflejo se ve más turbio, simulando una superficie menos pulida.

\section{Oclusión ambiental}

La oclusión ambiental es una técnica de rendering que se usa para calcular qué tan expuesto está cada punto de una escena a la luz ambiental.
\textit{Cone tracing} se puede usar para calcular este valor.
El efecto no aporta más al realismo de una escena una vez que se usan técnicas de iluminación indirecta, pero es un buen paso previo para ver el funcionamiento del algoritmo.

Para calcularlo, se lanzan varios conos, cubriendo el hemisferio centrado en la normal de la superficie en el punto.
El único valor necesario es la opacidad.
A medida que se viaja a lo largo de un cono, se va acumulando la opacidad de los vóxeles correspondientes.
Se define una distancia máxima y el criterio de parada es cuando el punto a lo largo del cono pasa esa distancia máxima, o la oclusión llega a 1.

\section{Conos de sombra}

De la misma manera que el trazado de rayos logra sombras lanzando un rayo hacia la fuente de luz, mientras que en cone tracing se logran lanzando un cono hacia la fuente de luz.
El cono toma en cuenta únicamente la opacidad y su criterio de parada es alcanzar la luz o 1 de opacidad antes.
El beneficio de que sea un cono en lugar de un rayo, y de tener la estructura jerárquica del \textit{octree}, es que se logran sombras suaves, sin necesidad de tener que tomar muchas muestras y promediarlas.

% END.


% implementación
\graphicspath{{chapters/4_implementación/figures/}}

\chapter{Implementación}\label{chap:implementation}

La implementación utilizó el lenguaje de programación Rust \cite{rust-lang} y la API de gráficos OpenGL \cite{opengl-spec}.

La arquitectura principal consta de tres paquetes: \textit{cli}, \textit{core} y \textit{engine}.
\textit{Engine} contiene todas las abstracciones sobre OpenGL utilizadas, provee tipos como \textit{Transform}, \textit{Light}, \textit{Camera} que permiten manipular objetos en el espacio 3D.
\textit{Core} contiene todos los algoritmos de voxel cone tracing que hemos mencionado: voxelización, construcción del SVO, filtrado, actualización y el trazado de conos en si.
El paquete \textit{cli} es el punto de entrada de la aplicación, procesa los argumentos pasados por linea de comandos y archivos de configuración y utiliza las funcionalidades expuestas por \textit{engine} y \textit{core} para crear un ambiente 3D con los efectos logrados por voxel cone tracing.
Estos tres paquetes se muestran en la figura \ref{fig:overall_architecture}.

\begin{figure}
    \centering
    \includegraphics[width=\textwidth]{arquitectura_vct.png}
    \caption{Arquitectura de la implementación}
    \label{fig:overall_architecture}
\end{figure}

En las siguientes secciones se verán a detalle las arquitecturas de cada uno de estos paquetes.

\section{Engine}

Engine expone una serie de abstracciones para interactuar con OpenGL.


\section{Core}

% TODO

\subsection{Representación del SVO}\label{implementation:svo_representation}

La forma de representar este árbol es con una textura lineal en GPU, conocida como la node pool.
Cada texel (píxel de textura) de esta textura es un puntero a otro nodo.
Se toma la convención de que cada grupo de 8 texels es un nodo, cada texel es un puntero al hijo correspondiente.
Los primeros 4 texels representan una subdivisión con valor de $z$ menor mientras que los últimos 4 representan una con valor de $z$ mayor.
Dentro de cada grupo de 4, los primeros 2 son un valor menor de $y$ y los últimos uno mayor.
Dentro de los grupos de 2, primero es menor $x$, último mayor.
Si un texel tiene el valor $0$, entonces ese hijo del nodo no existe.
Si un texel tiene un valor $x != 0$, entonces en la posición $x * 8$ comienza un nuevo nodo, que termina en $x * 8 + 7$.

Cada nodo tiene asociado un brick.
Los bricks son texturas 3D de $3^3$ texels que buscan aproximar la región de la escena que representa el nodo.
Se almacenan en una gran textura 3D llamada la brick pool, con lo cual cada brick es una región de $3^3$ texels de esta gran textura.
Mientras los nodos tienen únicamente punteros a sus hijos, los bricks son los que almacenan los atributos, como el color y la normal.
Cada texel dentro de un brick se llama vóxel, porque además de ser un píxel de textura, también es un píxel de volumen.
Es en estos vóxels que se almacenan los atributos con los que se va a trabajar.

\begin{figure}[h!]
    \centering
    \includegraphics[width=\textwidth]{node-pool-example.png}
    \caption{Ejemplo node pool y brick pool}
    \label{fig:node_pool_example}
\end{figure}

En la figura \ref{fig:node_pool_example} se muestra un ejemplo de node pool y brick pool.
En este ejemplo, vemos que la node pool está pintada de rojo, mientras que la brick pool está pintada de azul.
Cada nodo de la node pool está numerado en el margen izquierdo ($0, 1, 2$), está formado por $8$ texels (los pequeños números arriba de cada caja).
Los números dentro de cada texel de un nodo son los índices del nodo (que debe ser multiplicado por $8$ para conseguir el texel correspondiente) hijo.
Los texels que tienen $0$ indican que esa subdivisión no tiene un hijo.
Acá es donde se ve que la estructura es esparza.

Cada brick es una sección de $3^3$ texels de una gran textura 3D.
Cada nodo tiene asociado un brick, que es identificado únicamente por su índice.
Dado que los bricks existen en una textura 3D, la manera de identificarlos es con un vector de $\mathbb{R}^3$.
Para esto, se usa una función que convierte cada índice de $\mathbb{R}$ en un vector de $\mathbb{R}^3$.

Cada nodo podría tener asociado solo un valor escalar para el atributo que se quiera guardar en él, en lugar de asociarles una textura 3D (el brick).
Sin embargo, usando texturas podemos conseguir los beneficios de la interpolación trilineal provista por el hardware de la GPU al tomar un sample dentro del brick.
Si el sample cae entre dos texels, se interpola.
Esto permite mejorar la calidad de la imagen, como veremos en el capítulo \ref{chap:experiments}.

La node pool y brick pool son la forma en la que se almacenan en memoria los nodos y bricks respectivamente.
Si bien los nodos y los bricks se almacenan usando indices, la región de la escena que representan es totalmente distinta.
Los nodos representan secciones cada vez más chicas de la escena, que resultan de subdividir al nodo padre, ocupando el nodo con indice $0$ toda la escena.
Los bricks se ubican en la escena sobre sus nodos asociados, y se extienden un poco más hacia cada dirección.
Esto es porque los vóxels de los bricks están centrados en los vértices de los nodos.

% TODO: Tuve que mover fig:node_and_brick a diseño porque hay que hablar de nodos y bricks antes.

En la figura \ref{fig:node_and_brick} también se ve la razón por la que los bricks son de tamaño $3^3$, para poder cubrir el espacio del nodo pero también extenderse un poco más allá de su límite.
Esto es necesario para poder conseguir valores de más allá del espacio del nodo y traerlos al brick, así luego, al interpolar los valores dentro del brick, se están teniendo en cuenta valores de una región más grande.
El resultado es que dos nodos vecinos comparten una frontera de vóxels de sus bricks asociados.
Esto se muestra en la figura \ref{fig:brick_border_overlap}.
Dado que estos vóxels representan la misma región del espacio, es necesario que sus valores sean los mismos.
Si bien en memoria son dos entidades distintas que existen en distintos lugares, espacialmente son las mismas, con lo cual debe asegurarse que sus valores sean iguales.

\subsection{Vóxels anisotrópicos}\label{implementation:anisotropic_vóxels}

El cálculo del filtrado anisotrópico requiere conocimiento de los bricks vecinos a la hora de calcular el valor de un vóxel limítrofe.
El shader que realiza el filtrado anisotrópico ejecuta un hilo por brick, con lo cual no se tiene acceso a los bricks vecinos.

Para resolver esto, se carga el vecino en la dirección de filtrado para poder completar los valores direccionales.
Luego se usa un shader que promedia los valores direccionales con los vecinos de las otras dos direcciones.

\subsection{Nodos frontera}\label{implementation:border_nodes}

El procedimiento para producir los nodos frontera reutiliza lo ya existente para generar nodos dado una lista de vóxel fragments\ref{chap:design}, en lugar de modificar directamente con la node pool\ref{chap:design} ya existente.
Al proceso de construcción del octree se le agregan dos pasos:

\begin{itemize}
    \item Construír lista de vóxel fragments
    \item Construir nodos del octree a partir de la lista de vóxel fragments
    \item NUEVO: Construír una lista de vóxel fragments frontera a partir de la lista original de vóxel fragments
    \item NUEVO: Agregar nodos frontera al octree usando la lista de vóxel fragments frontera
\end{itemize}

Se toma la lista de vóxel fragments que fue generada a partir de la geometría de la escena y se hace pasar por un nuevo shader.
Este nuevo shader toma esta lista de vóxel fragments y crea una nueva lista de vóxel fragments ``ficticia'', porque no son parte de la geometría.
Por cada vóxel fragment de la lista original, se recorre el octree y, por cada una de las seis direcciones principales (X, -X, Y, -Y, Z, -Z), se busca si tiene un nodo vecino.
En el caso que no lo tenga, se agrega a la nueva lista un vóxel fragment cuya posición es la del original desplazada hacia la dirección considerada, para que al crearse el nodo en la estructura, este tome el lugar del vecino faltante.

Aprovechar este shader logra que solo sea necesario crear nodos para el último nivel y se subdividen nodos de niveles anteriores automáticamente a medida que se desciende por el árbol.

Una vez se obtiene la nueva lista de vóxel fragments, esta se hace pasar por el mismo shader que convierte vóxel fragments en nodos del octree.
Estos nuevos nodos se agregan al final del buffer de la estructura.
Dado que los nodos nuevos están en distintos niveles, se mantiene una lista de índices distinta para mantener de dónde a dónde va un nivel N en el buffer.

% TODO: Agregar diagramita de como se agregan los nodos frontera al final y se mantienen dos buffers de indices.

\subsection{Menú}

A lo largo de la implementación, fue necesario depurar varios errores y correr pruebas.
Para esto, fue muy útil contar con una interfaz gráfica o menú para seleccionar varias opciones y poder ver valores de la GPU en tiempo real.
Se desarrolló este menú utilizando un paquete del ecosistema de Rust llamado Egui \cite{egui}.

Egui permite rápidamente crear una interfaz gráfica basada en ventanas que se renderiza junto con la aplicación en cada frame.
Es muy sencillo conectar valores del código a etiquetas en el menú y botónes en el menú a acciones.

La herramienta más importante del menú a la hora de depurar fue una ventana que muestra todos los nodos de la node pool.
Esta se puede ver en la figura \ref{fig:node_positions_menu}.
En el menú se muestran todos los nodos de la node pool con su indice (0, 1, 2, ...) y sus coordenadas dentro de la escena (entre 0 y 255).
Al apretar cualquier nodo, se muestra en la escena un cubo delimitando la región de la escena representada por ese nodo.
Se pueden tener muchos nodos mostrándose a la vez.

\begin{figure}
    \centering
    \includegraphics[width=.25\textwidth]{node_positions_menu.png}
    \caption{Menú de nodos}
    \label{fig:node_positions_menu}
\end{figure}

Este menú se puede filtrar, tanto por índice como por coordenadas.
También se pueden mostrar los vecinos de cada nodo y los bricks, que para no llenar la pantalla mucho, se pueden mostrar individualmente cada capa del brick, con $z=0,1,2$.

También se usó la interfaz gráfica para reportar los FPS que fueron usados en el capítulo \ref{chap:experiments} y mostrar otros valores en pantalla, relevantes para el algoritmo.

\section{Cli}

\subsection{Archivos de escena}

Para poder fácilmente cargar distintos modelos y probar el algoritmo en ellos, se creó un formato de archivos de escena.
Estos archivos usan extensión RON, por Rusty Object Notation, una notación similar a JSON, JavaScript Object Notation, pero para Rust.

% TODO: Agregar ejemplos de archivos y qué renderizan

\section{Herramientas} % TODO: Porbablemente buscar de meter esto en otra sección o en la intro del capítulo

\subsection{Rust}

Se eligió el lenguaje de programación Rust para la implementación del algoritmo debido a varias razones:
\begin{itemize}
    \item Es rápido
    \item Tiene un manejador de paquetes por defecto con muchas herramientas disponibles
    \item Tiene una comunidad muy activa y buena documentación
\end{itemize}

Obviamente, se consideró C++ dado que es el lenguaje más popular para aplicaciones gráficas.
El factor que más pesó en la decisión fue la existencia de un manejador de paquetes por defecto.
Esto permitió instalar varias dependencias mucho más rápido y cambiarlas a lo largo del ciclo de desarrollo.

\subsection{OpenGL}

Como se mencionó en el capítulo \ref{chap:design}, el algoritmo fue mayoritariamente implementado en la GPU.
Si bien el lenguaje de programación de la CPU es importante, la elección de la API de gráficos fue, sin duda, la más importante.
Se consideraron Vulkan y CUDA pero se terminó optando por OpenGL debido a varios factores:
\begin{itemize}
    \item Facilidad de desarrollo
    \item Conocimiento previo
    \item Facilidad de trabajo con compute shaders
\end{itemize}

Si bien Vulkan es una API más moderna, el equipo del presente trabajo no contaba con la suficiente experiencia como para utilizarlo adecuadamente, lo cual hubiera llevado a mucho tiempo desperdiciado aprendiendo la herramienta.
CUDA es usualmente utilizado para cálculos arbitrarios en la GPU pero debido a la necesidad de también usar el pipeline gráfico, se optó por usar los compute shaders de OpenGL.


% experimentación
\graphicspath{{chapters/5_experimentos/figures/}}

\chapter{Experimentación}\label{chap:experiments}

Para analizar el rendimiento de la implementación, se realizaron experimentos en distintas tarjetas gráficas.
La escena utilizada se muestra en la figura \ref{fig:cornell-box-full}.
Esta escena presenta luz indirecta difusa, luz indirecta especular y sombras suaves.

\begin{figure}
	\centering
	\includegraphics[width=\textwidth]{cornell-box-full.png}
	\caption{Escena con iluminación indirecta mediante \textit{voxel cone tracing}}
	\label{fig:cornell-box-full}
\end{figure}

Los experimentos varian la cantidad de vóxeles por dimensión, y registran los cuadros por segundo alcanzados (FPS) y el tiempo de construcción del \textit{octree} en segundos.
Para conseguir los FPS se corrió el programa por un minuto, tomando una muestra de este valor cada segundo, luego se promediaron.
Para el tiempo de construcción del \textit{octree}, se construyó 50 veces y se promedió el tiempo.
Para que el cache no afectara estas ejecuciones en serie, se realizaron en procesos separados.

En las tablas \ref{tab:cisco-laptop}, \ref{tab:pizzo-laptop} y \ref{tab:pizzo-desktop} se muestran los resultados de estos experimentos en tarjetas Intel Mesa ADL GT2, ??? y Nvidia RTX 4070 respectivamente.
Una observacion es que la implementación parece no estar bien optimizada, dado que en las tarjetas gráficas de laptops en las que se ejecutó, se consiguieron muy pocos cuadros por segundo y tiempos muy lentos de construcción del octree.
Los valores de estas métricas son mucho menores que en las del artículo original del algoritmo, en el que se mostraron entre 20 y 30 cuadros por segundo y 280 milisegundos de construcción del \textit{octree}, en lugar de segundos.

Otra observación es que la tarjeta Nvidia RTX 4070 trae excelentes resultados en terminos de cuadros por segundo, pero sorprendentemente es más lenta que la Intel Mesa ADL GT2 en construír el \textit{octree}.
Esto puede deberse a que la primer tarjeta posee mucho más paralelización, por lo que puede realizar todos los trazados de conos sin problemas, pero este no es el cuello de botella para la construcción del \textit{octree}.
En la construcción, probablemente lo que más este beneficiando a la segunda tarjeta es la cercanía física entre CPU y GPU, que implica una comunicación más rápida.

\begin{table}
\centering
\begin{tabular}{|c|c|c|}
	\hline
	\textbf{Vóxeles} & \textbf{FPS} & \textbf{Construcción del \textit{octree}} \\
	\hline
	$256$ & $14.470$ & $1.451$ \\
	\hline
	$512$ & $11.301$ & $1.490$ \\
	\hline
	$1024$ & $9.215$ & $1.656$ \\
	\hline
\end{tabular}
\caption{Experimentos para una Intel Mesa ADL GT2}
\label{tab:cisco-laptop}
\end{table}

\begin{table}
\centering
\begin{tabular}{|c|c|c|}
	\hline
	\textbf{Vóxeles} & \textbf{FPS} & \textbf{Construcción del \textit{octree}} \\
	\hline
	$256$ & $12.632$ & $2.461$ \\
	\hline
	$512$ & $10.363$ & $2.780$ \\
	\hline
	$1024$ & $8.825$ & $3.245$ \\
	\hline
\end{tabular}
\caption{Experimentos para una ???}
\label{tab:pizzo-laptop}
\end{table}

\begin{table}
\centering
\begin{tabular}{|c|c|c|}
	\hline
	\textbf{Vóxeles} & \textbf{FPS} & \textbf{Construcción del \textit{octree}} \\
	\hline
	$256$ & $141.570$ & $1.949$ \\
	\hline
	$512$ & $141.496$ & $1.936$ \\
	\hline
	$1024$ & $141.406$ & $1.950$ \\
	\hline
\end{tabular}
\caption{Experimentos para una RTX 4070}
\label{tab:pizzo-desktop}
\end{table}


%conclusiones
\graphicspath{{chapters/6_conclusión/figures/}}

% En este capítulo se evalúan los resultados alcanzados y
% dificultades encontradas, se establece lo que se planteó hacer y lo que se hizo
% realmente, cuales fueron los aportes, se muestran posibles extensiones al trabajo, se
% realiza una autocrítica de lo que se hizo y lo que faltó (por problemas de tiempo,
% recursos, cómo se puede continuar, qué cosas hacer, prioridades, etc.) y se incluye
% información sobre la gestión del proyecto, si aplica

%%%%%%%%%%%%%%%%%%%%%%%%%%%%%%%%%%%%%%%%%%%%%%%%%%%%%%%%%%%%%%%%%%%%%%%%%%%%%%%%%%%%%%%%%%%%%%%%
% Parte final
%%%%%%%%%%%%%%%%%%%%%%%%%%%%%%%%%%%%%%%%%%%%%%%%%%%%%%%%%%%%%%%%%%%%%%%%%%%%%%%%%%%%%%%%%%%%%%%%

\chapter{Conclusiones y Trabajo Futuro}\label{chap:conclusions}

En este proyecto se logró una implementación práctica y funcional de la información provista por un artículo de revista \cite{voxel-cone-tracing} y una tesis \cite{gigavoxels}.
El objetivo principal fue estudiar y comprender a profundidad el algoritmo de Voxel Cone Tracing, enfrentando los retos de llevarlo a un entorno funcional y documentando el proceso de manera que sirva como recurso útil para otros interesados en este tema.
Así, se cumplieron los objetivos principales planteados al inicio del trabajo.

\section{Principales conclusions}

Se consiguió implementar el algoritmo de Voxel Cone Tracing.
Con el mismo se logró generar un rendering, en el hardware intermedio probado y en una escena compleja, con valores de cuadros por segundo cercanos al tiempo real.

A lo largo del proyecto, se profundizó en el entendimiento del algoritmo y en los desafíos del desarrollo en GPU.
Se dejó un recurso, detallando el algoritmo paso a paso, y el codigo queda abierto como ejemplo para futuros desarrolladores.

Este proyecto no solo permitió desarrollar un código concreto, sino que también sirvió como ejercicio para materializar conceptos teóricos en aplicaciones reales.


El desarrollo del proyecto presentó diversos desafíos.
Programar en la GPU implicó un cambio de paradigma respecto al modelo tradicional en CPU.
Las restricciones arquitectónicas y el diseño del pipeline de cómputo representaron una curva de aprendizaje significativa.
La depuración en GPU, limitada por la falta de herramientas tradicionales, requirió el desarrollo de utilidades personalizadas para visualizar los estados intermedios del algoritmo y diagnosticar problemas.

\section{Trabajo futuro}

El proyecto presenta varias oportunidades para mejorar y ampliar la implementación actual:

\begin{itemize}
    \item Soporte para objetos dinámicos: Implementar el manejo de objetos dinámicos en la escena y actualizar el octree en tiempo real, una característica considerada pero no incluida en este trabajo.
    \item Migración a Vulkan: Portar la implementación a Vulkan para aprovechar un control más detallado de la GPU, potencialmente mejorando el rendimiento y la flexibilidad.
    \item Mejora de herramientas de visualización: Desarrollar herramientas más avanzadas para inspeccionar los estados intermedios del algoritmo, facilitando la depuración y la investigación.
\end{itemize}

Los resultados y experiencias obtenidos proporcionan una base sólida para futuros desarrollos de Voxel Cone Tracing y otras técnicas de iluminación global.

\backmatter % no numerar capítulos 

% referencias bibliográficas

\newpage

\printbibliography

%%%%%%%%%%%%%%%%%%%%%%%%%%%%%%%%%%%%%%%%%%%%%%%%%%%%%%%%%%%%%%%%%%%%%%%%%%%%%%%%%%%%%%%%%%%%%%%%

% anexos

\begin{appendix}

% Manual de usuario
\include{chapters/annex_1.tex}

\end{appendix}

%%%%%%%%%%%%%%%%%%%%%%%%%%%%%%%%%%%%%%%%%%%%%%%%%%%%%%%%%%%%%%%%%%%%%%%%%%%%%%%%%%%%%%%%%%%%%%%%

\end{document}
